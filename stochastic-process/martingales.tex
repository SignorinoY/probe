\chapter{Martingales}

\section{Conditional Expectation}

\begin{definition}[Conditional Expectation]

\end{definition}

\begin{example}
    \begin{enumerate}
        \item
    \end{enumerate}
\end{example}

\begin{property}
    
\end{property}

\section{Martingales}

Let $\mathcal{F}_{n}$ be a filtration, i.e., an increasing sequence of $\sigma$-fields.
\begin{definition}[Martingale]
    A sequence $\left\{X_{n}\right\}$ of real-valued random variables  is said to be a martingale with respect to $\mathcal{F}_{n}$, if
    \begin{enumerate}
        \item $X_{n}$ is integrable, i.e., $E\left|X_{n}\right|<\infty$
        \item $X_{n}$ is adapted to $\mathcal{F}_{n}$, i.e., $\forall n,X_{n}\in \mathcal{F}_{n}$
        \item $X_{n}$ satisfies the martingale condition, i.e.,
              \begin{equation}
                  E\left(X_{n+1}\mid\mathcal{F}_{n}\right)=X_{n},\quad\forall n
              \end{equation}
    \end{enumerate}
\end{definition}

\begin{note}
    If in the last definition $=$ is replaced by $\leq$ or $\geq$, then $X$ is said to be a supermartingale or submartingale, respectively.
\end{note}

\begin{example}[ Linear Martingale]
    
\end{example}

\begin{example}[ Quadratic Martingale]
    
\end{example}

\begin{example}[ Random Walk]
    Suppose $X_{n}=X_{0}+\xi_{1}+\cdots+\xi_{n}$, where $X_{0}$ is constant, $\xi_{m}$ are independent and have $E\xi_{m}=0,\sigma_{m}^{2}=E\xi_{m}^{2}<\infty$. Let $\mathcal{F}_{n}=\sigma\left(\xi_{1},\ldots,\xi_{n}\right)$ for $n\geq 1$ and take $\mathcal{F}_{0}=\{\emptyset, \Omega\}$. Show $X_{n}$ is a martingale, and $X_{n}^{2}$ is a submartingale.
\end{example}

\begin{proof}
    It is obvious that,
    \begin{equation*}
        E\left|X_{n}\right|<\infty,\quad X_{n}\in\mathcal{F}_{n}
    \end{equation*}
    Since $\xi_{n+1}$ is independent of $\mathcal{F}_{n}$, so using the linearity of conditional expectation, (4.1.1), and Example 4.1.4,
    \begin{equation*}
        E\left(X_{n+1}\mid\mathcal{F}_{n}\right)=E\left(X_{n}\mid\mathcal{F}_{n}\right)+E\left(\xi_{n+1}\mid\mathcal{F}_{n}\right)=X_{n}+E\xi_{n+1}=X_{n}
    \end{equation*}
    So $X_{n}$ is a martingale, and Theorem 4.2.6 implies $X_{n}^{2}$ is a submartingale.
\end{proof}

\begin{note}
    If we let $\lambda=x^{2}$ and apply Theorem 4.4.2 to $X_{n}^{2}$, we get Kolmogorov's maximal inequality, Theorem 2.5.5:
    \begin{equation}
        P\left(\max_{1\leq m\leq n}\left|X_{m}\right|\geq x\right)\leq x^{-2}\operatorname{var}\left(X_{n}\right)
    \end{equation}
\end{note}

\section{Doob's Inequality}

\begin{theorem}[Doob's Decomposition]

\end{theorem}

\begin{theorem}[Doob's Inequality]

\end{theorem}

\begin{theorem}[$L^{p}$ Maximum Inequality]

\end{theorem}

\section{Uniform Integrability}

\section{Optional Stopping Theorems}
