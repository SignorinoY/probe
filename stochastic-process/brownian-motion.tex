\chapter{Brownian Motion}

\begin{definition}[Brownian Motion (1)]
    A real-valued stochastic process $B(t),t\geq 0$ is a Brownian motion, if
    \begin{enumerate}
        \item for any $0=t_{0}\leq t_{1}\leq\ldots\leq t_{n}$ the increments
              \begin{equation*}
                  B\left(t_{1}\right)-B\left(t_{0}\right),\ldots,B\left(t_{n}\right)-B\left(t_{n-1}\right)
              \end{equation*}
              are independent
        \item for any $s,t\geq 0$ and Borel sets $A\in\mathbb{R}$,
              \begin{equation}
                  P\left(B(s+t)-B(s)\in A\right)=\int_{A}(2\pi t)^{-1/2}\exp\left(-x^{2}/2t\right)\,\mathrm{d}x
              \end{equation}
        \item the sample paths $t\rightarrow B(t)$ are a.s. continuous
    \end{enumerate}
\end{definition}

The one-dimensional Brownian motion has the following properties

\begin{property}
    Suppose $B(0)=0$, then we have
    \begin{enumerate}
        \item $EB_{t}=0,\operatorname{Var}\left(B_{t}\right)=t,\quad t\geq 0$.
        \item $\operatorname{Cov}\left(B_{s},B_{t}\right)=s,\operatorname{Corr}\left(B_{s},B_{t}\right)=\sqrt{\frac{s}{t}},\quad\forall 0\leq s\leq t$.
    \end{enumerate}
\end{property}

\begin{proof}
    \begin{enumerate}
        \item Since $B_{t}=B_{t}-B_{0}\sim N(0, t)$, then we have
              \begin{equation*}
                  EB_{t}=0,\operatorname{Var}\left(B_{t}\right)=t
              \end{equation*}
        \item Suppose $0\leq s\leq t$,
              \begin{equation*}
                  \operatorname{Cov}\left(B_{s}, B_{t}\right)=E\left[\left(B_{s}-EB_{s}\right)\left(B_{t}-EB_{t}\right)\right]=EB_{s}B_{t}
              \end{equation*}
              Let $B_{t}=\left(B_{t}-B_{s}\right)+B_{s}$, we have
              \begin{equation*}
                  \begin{aligned}
                      EB_{s}B_{t} & =E\left[B_{s}\cdot\left(\left(B_{t}-B_{s}\right)+B_{s}\right)\right] \\
                                  & =E\left[B_{s}\cdot\left(B_{t}-B_{s}\right)\right]+EB_{s}^{2}
                  \end{aligned}
              \end{equation*}
              Since $B_{s}=B_{s}-B_{0}$ and $B_{t}-B_{s}$ are independent,
              \begin{equation*}
                  E\left[B_{s} \cdot\left(B_{t}-B_{s}\right)\right]=EB_{s} \cdot E\left[B_{t}-B_{s}\right]=0
              \end{equation*}
              Thus
              \begin{equation*}
                  \operatorname{Cov}\left(B_{s}, B_{t}\right)=EB_{s}^{2}=s
              \end{equation*}
              And
              \begin{equation*}
                  \operatorname{Corr}\left(B_{s},B_{t}\right)=\frac{\operatorname{Cov}\left(B_{s},B_{t}\right)}{\sigma_{B_{s}}\sigma_{B_{t}}}=\frac{s}{\sqrt{st}}=\sqrt{\frac{s}{t}}
              \end{equation*}
    \end{enumerate}
\end{proof}

A second equivalent definition of Brownian motion are as followed,

\begin{definition}[Brownian Motion (2)]
    A real-valued stochastic process $B(t),t\geq 0$, \textbf{starting from $0$}, is a Brownian motion, if
    \begin{enumerate}
        \item $B(t)$ is a Gaussian process\footnote{Gaussian process, i.e., all its finite dimensional distributions are multivariate normal.}
        \item $\forall s,t\geq 0,EB_{s}=0$ and $EB_{s}B_{t}=s\wedge t$
        \item the sample paths $t\rightarrow B(t)$ are a.s. continuous
    \end{enumerate}
\end{definition}

\section{Markov Properties}

\section{Martingales}

\begin{example}
    Suppose $B_{t}$ is a Brownian motion, then $B_{t}^{2}-t$ is a martingale.
\end{example}

\begin{proof}
    Let $B_{t}^{2}=\left(B_{s}+B_{t}-B_{s}\right)^{2}$, we have
    \begin{equation*}
        \begin{aligned}
            E_{x}\left(B_{t}^{2}\mid\mathcal{F}_{s}\right) & =E_{x}\left(B_{s}^{2}+2 B_{s}\left(B_{t}-B_{s}\right)+\left(B_{t}-B_{s}\right)^{2} \mid \mathcal{F}_{s}\right)                            \\
                                                           & =B_{s}^{2}+2 B_{s} E_{x}\left(B_{t}-B_{s} \mid \mathcal{F}_{s}\right)+E_{x}\left(\left(B_{t}-B_{s}\right)^{2} \mid \mathcal{F}_{s}\right) \\
                                                           & =B_{s}^{2}+0+(t-s)
        \end{aligned}
    \end{equation*}
    since $B_{t}-B_{s}$ is independent of $\mathcal{F}_{s}$ and has mean 0 and variance $t-s$.
\end{proof}

\begin{example}
    Suppose $B_{t}$ is a Brownian motion, $\exp\left(\theta B_{t}-\left(\theta^{2}t/2\right)\right)$ is a martingale.
\end{example}

\begin{proof}
    Let $B_{t}=B_{t}-B_{s}+B_{s}$, then
    \begin{equation*}
        \begin{aligned}
            E_{x}\left(\exp\left(\theta B_{t}\right)\mid\mathcal{F}_{s}\right) & =\exp \left(\theta B_{s}\right)E\left(\exp\left(\theta\left(B_{t}-B_{s}\right)\right)\mid\mathcal{F}_{s}\right) \\
                                                                               & =\exp\left(\theta B_{s}\right)\exp\left(\theta^{2}(t-s)/2\right)
        \end{aligned}
    \end{equation*}
    since $B_{t}-B_{s}$ is independent of $\mathcal{F}_{s}$ and has mean 0 and variance $t-s$. Thus
    \begin{equation*}
        \begin{aligned}
            E_{x}\left(\exp\left(\theta B_{t}-\left(\theta^{2}t/2\right)\right)\mid\mathcal{F}_{s}\right) & =E_{x}\left(\exp\left(\theta B_{t}\right)\mid\mathcal{F}_{s}\right)\cdot\exp\left(-\left(\theta^{2}t/2\right)\right) \\
                                                                                                          & =\exp\left(\theta B_{s}-\left(\theta^{2}s/2\right)\right)
        \end{aligned}
    \end{equation*}
\end{proof}

\begin{theorem}[Lévy's martingale characterization]
    Let $B(t),t\geq 0$, be a real-valued stochastic process and let $\mathcal{F}_{t}=\sigma\left(B_{s},s\leq t\right)$ be the filtration generated by it. Then $B(t)$ is a Brownian motion if and only if
    \begin{enumerate}
        \item $B(0)=0$ a.s.
        \item the sample paths $t\rightarrow B(t)$ are continuous a.s.
        \item $B(t)$ is a martingale with respect to $\mathcal{F}_{t}$
        \item $|B(t)|^{2}-t$ is a martingale with respect to $\mathcal{F}_{t}$
    \end{enumerate}
\end{theorem}

\section{Sample Paths}

Let $0=t_{0}^{n}<t_{1}^{n}<\cdots<t_{n}^{n}=T$, where $t_{i}^{n}=\frac{iT}{n}$ be a partition of the interval $[0,T]$ into $n$ equal parts, and
\begin{equation}
    \Delta_{i}^{n}B=B\left(t_{i+1}^{n}\right)-B\left(t_{i}^{n}\right)
\end{equation}
be the corresponding increments of the Brownian motion $B(t)$.

\begin{theorem} \label{thm:limited-square-variation}
    \begin{equation}
        \lim_{n\rightarrow\infty}\sum_{i=0}^{n-1}\left(\Delta_{i}^{n}B\right)^{2}=T\quad\text { in }\quad L^{2}
    \end{equation}
\end{theorem}

\begin{proof}
    Since the increments $\Delta_{i}^{n}B$ are independent and
    \begin{equation*}
        E\left(\Delta_{i}^{n}B\right)=0,\quad E\left(\left(\Delta_{i}^{n}B\right)^{2}\right)=\frac{T}{n},\quad E\left(\left(\Delta_{i}^{n}B\right)^{4}\right)=\frac{3T^{2}}{n^{2}}
    \end{equation*}
    it follows that
    \begin{equation*}
        \begin{aligned}
            E\left(\left[\sum_{i=0}^{n-1}\left(\Delta_{i}^{n}B\right)^{2}-T\right]^{2}\right)= & E\left(\left[\sum_{i=0}^{n-1}\left(\left(\Delta_{i}^{n}B\right)^{2}-\frac{T}{n}\right)\right]^{2}\right)                                                   \\
            =                                                                                  & \sum_{i=0}^{n-1}E\left[\left(\left(\Delta_{i}^{n}B\right)^{2}-\frac{T}{n}\right)^{2}\right]                                                                \\
            =                                                                                  & \sum_{i=0}^{n-1}\left[E\left(\left(\Delta_{i}^{n}B\right)^{4}\right)-\frac{2T}{n}E\left(\left(\Delta_{i}^{n}B\right)^{2}\right)+\frac{T^{2}}{n^{2}}\right] \\
            =                                                                                  & \sum_{i=0}^{n-1}\left[\frac{3T^{2}}{n^{2}}-\frac{2T^{2}}{n^{2}}+\frac{T^{2}}{n^{2}}\right]                                                                 \\
            =                                                                                  & \frac{2T^{2}}{n}\rightarrow 0,\quad n\rightarrow\infty
        \end{aligned}
    \end{equation*}
\end{proof}

\begin{definition}[Variation]
    The variation of a function $f:[0,T]\rightarrow\mathbb{R}$ is defined to be
    \begin{equation}
        \limsup_{\Delta t\rightarrow 0}\sum_{i=0}^{n-1}\left|f\left(t_{i+1}\right)-f\left(t_{i}\right)\right|
    \end{equation}
    where $t=\left(t_{0},t_{1},\ldots,t_{n}\right)$ is a partition of $[0,T]$, i.e. $0=t_{0}<t_{1}<\cdots<t_{n}=T$, and where
    \begin{equation}
        \Delta t=\max_{i=0,\ldots,n-1}\left|t_{i+1}-t_{i}\right|
    \end{equation}
\end{definition}

\begin{theorem}
    The variation of the paths of $B(t)$ is infinite a.s..
\end{theorem}

\begin{proof}
    Consider the sequence of partitions $t^{n}=\left(t_{0}^{n},t_{1}^{n},\ldots,t_{n}^{n}\right)$ of $[0,T]$ into $n$ equal parts. Then
    \begin{equation*}
        \sum_{i=0}^{n-1}\left|\Delta_{i}^{n}B\right|^{2}\leq\left(\max_{i=0,\ldots,n-1}\left|\Delta_{i}^{n}B\right|\right)\sum_{i=0}^{n-1}\left|\Delta_{i}^{n}B\right|
    \end{equation*}

    Since the paths of $B(t)$ are a.s. continuous on $[0,T]$,
    \begin{equation*}
        \lim_{n\rightarrow\infty}\left(\max_{i=0,\ldots,n-1}\left|\Delta_{i}^{n}B\right|\right)=0\quad\text{ a.s. }
    \end{equation*}

    By Theorem \ref{thm:limited-square-variation}, we have
    \begin{equation*}
        \lim_{n\rightarrow\infty}\sum_{i=0}^{n-1}\left(\Delta_{i}^{n}B\right)^{2}=T\quad\text { in }\quad L^{2}
    \end{equation*}
    Since every sequence of random variables convergent in $L^{2}$ has a subsequence convergent a.s. There is a subsequence $t^{n_{k}}=\left(t_{0}^{n_{k}},t_{1}^{n_{k}},\ldots,t_{n_{k}}^{n_{k}}\right)$ of partitions such that
    \begin{equation*}
        \lim_{k\rightarrow\infty}\sum_{i=0}^{n_{k}-1}\left|\Delta_{i}^{n_{k}}B\right|^{2}=T\quad\text{ a.s. }
    \end{equation*}

    Since
    \begin{equation*}
        \sum_{i=0}^{n_{k}-1}\left|\Delta_{i}^{n_{k}}B\right|\geq\frac{\sum_{i=0}^{n_{k}-1}\left|\Delta_{i}^{n_{k}}B\right|^{2}}{\max_{i=0,\ldots,n_{k}-1}\left|\Delta_{i}^{n_{k}}B\right|}
    \end{equation*}
    hence,
    \begin{equation*}
        \lim_{k\rightarrow\infty}\sum_{i=0}^{n_{k}-1}\left|\Delta_{i}^{n_{k}}B\right|=\infty\quad\text{ a.s. }
    \end{equation*}
    while
    $$
        \lim _{k \rightarrow \infty} \Delta t^{n_{k}}=\lim _{k \rightarrow \infty} \frac{T}{n_{k}}=0
    $$
\end{proof}

\section{It\^o Stochastic Calculus}

\begin{definition}[Random Step Process]

\end{definition}

The following properties hold

\begin{property}
    For $\forall f,g\in M_{t}^{2}$, $\forall\alpha,\beta\in\mathbb{R}$, and $\forall 0\leq s<t$ :
    \begin{enumerate}
        \item Linearity:
              \begin{equation}
                  \int_{0}^{t}(\alpha f(r)+\beta g(r))\,\mathrm{d}B(r)=\alpha \int_{0}^{t}f(r)\,\mathrm{d}B(r)+\beta\int_{0}^{t}g(r)\,\mathrm{d}B(r)
              \end{equation}
        \item Isometry:
              \begin{equation}
                  E\left(\left|\int_{0}^{t}f(r)\,\mathrm{d}B(r)\right|^{2}\right)=E\left(\int_{0}^{t}|f(r)|^{2}\,\mathrm{d}r\right)
              \end{equation}
        \item Martingale Property:
              \begin{equation}
                  E\left(\int_{0}^{t}f(r)\,\mathrm{d}B(r)\mid\mathcal{F}_{s}\right)=\int_{0}^{s}f(r)\,\mathrm{d}B(r)
              \end{equation}
    \end{enumerate}
\end{property}

\begin{proof}

\end{proof}

\begin{definition}[Itô Process]
    A stochastic process $\xi(t),t\geq 0$ is called an Itô process if it has a.s. continuous paths and can be represented as
    \begin{equation}
        \xi(T)=\xi(0)+\int_{0}^{T}a(t)\,\mathrm{d}t+\int_{0}^{T}b(t)\,\mathrm{d}B(t)\quad\text { a.s. }
    \end{equation}
    where $b(t)$ is a process belonging to $M_{T}^{2}$ for all $T>0$ and $a(t)$ is a process adapted to the filtration $\mathcal{F}_{t}$ such that
    \begin{equation}
        \int_{0}^{T}|a(t)|\,\mathrm{d}t<\infty\quad\text { a.s. } \label{eq:condition-of-process-adapted-to-filtration}
    \end{equation}
    for all $T\geq 0$.
    The Itô process is denoted by
    \begin{equation}
        \mathrm{d}\xi(t)=a(t)\,\mathrm{d}t+b(t)\,\mathrm{d}B(t)
    \end{equation}
\end{definition}

\begin{remark}
    The class of all adapted processes $a(t)$ satisfying \ref{eq:condition-of-process-adapted-to-filtration} for some $T>0$ will be denoted by $\mathcal{L}_{T}^{1}$.
\end{remark}

\begin{theorem}[It\^o Formula]
    Suppose that $F(t,x)$ is a real-valued function with continuous partial derivatives $F_{t}^{\prime}(t,x),F_{x}^{\prime}(t,x)$ and $F_{xx}^{\prime\prime}(t,x)$ for all $t\geq 0$ and $x\in\mathbb{R}$ and $\xi(t)$ be an It\^o process. Assume the process $b(t)F_{x}^{\prime}(t,\xi(t))$ belongs to $M_{T}^{2}$ for all $T\geq 0$. Then $F(t,\xi(t))$ is an It\^o process such that
    \begin{equation}
        \begin{aligned}
            \mathrm{d}F(t,\xi(t))= & \left(F_{t}^{\prime}(t,\xi(t))+F_{x}^{\prime}(t,\xi(t))a(t)+\frac{1}{2}F_{xx}^{\prime\prime}(t,\xi(t))b(t)^{2}\right)\,\mathrm{d}t \\
                                   & +F_{x}^{\prime}(t,\xi(t))b(t)\,\mathrm{d}B(t)
        \end{aligned}
    \end{equation}
\end{theorem}

\begin{example}

\end{example}

\begin{example}

\end{example}

\begin{example}

\end{example}