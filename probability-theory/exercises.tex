\chapter{Exercises for Probability Theory and Examples}

\section{Measure Theory}

\begin{exercise}
    \begin{enumerate}
        \item Show that if $\mathcal{F}_{1}\subset \mathcal{F}_{2}\subset\ldots$ are $\sigma$ -algebras, then $\cup_{i}\mathcal{F}_{i}$ is an algebra.
        \item Give an example to show that $\cup_{i}\mathcal{F}_{i}$ need not be a $\sigma$ -algebra.
    \end{enumerate}
\end{exercise}

\begin{solution}
    \begin{enumerate}
        \item
        \textbf{Complement}: Suppose $A\in\cup_{i}\mathcal{F}_{i}$, since $\mathcal{F}_{1}\subset \mathcal{F}_{2}\subset\ldots$, assume $A\in\mathcal{F}_{i}$. And each $\mathcal{F}_{i}$ is $\sigma$-algebra,
        \begin{equation*}
            A^c\in\mathcal{F}_{i}\subset\cup_{i}\mathcal{F}_{i}.
        \end{equation*}
        \textbf{Finite Union}: Suppose $A_1,A_2\in\cup_{i}\mathcal{F}_{i}$, since $\mathcal{F}_{1}\subset \mathcal{F}_{2}\subset\ldots$, assume $A_1\in\mathcal{F}_{i},A_2\in\mathcal{F}_{j}$, such that,
        \begin{equation*}
            A_1,A_2\in\mathcal{F}_{\max(i,j)}.
        \end{equation*}
        Since each $\mathcal{F}_{i}$ is $\sigma$-algebra,
        \begin{equation*}
            A_1\cup A_2\in\mathcal{F}_{i}\subset\cup_{i}\mathcal{F}_{i}.
        \end{equation*}
        \item
        Let $\mathcal{F}_{i}$ be a Borel Set of $[1,2-\frac{1}{i}]$. Suppose $A_i=[1,2-\frac{1}{i}]\in\mathcal{F}_{i}$,
        \begin{equation*}
            \cup_{i}A_{i}=[1,2)\notin\cup_{i}\mathcal{F}_{i}.
        \end{equation*}
    \end{enumerate}
\end{solution}

\section{Laws of Large Numbers}

\section{Central Limit Theorems}

\begin{exercise}
    Let $g\geq 0$ be continuous. If $X_{n}\stackrel{d}{\rightarrow}X_{\infty},$ then
    \begin{equation*}
        \liminf_{n\rightarrow\infty}Eg\left(X_{n}\right)\geq Eg\left(X_{\infty}\right).
    \end{equation*}
    \label{ex:fatou-lemma-distribution}
\end{exercise}

\begin{solution}
    Let $Y_n\stackrel{d}{=}X_n,1\leq n\leq\infty$ with $Y_n\stackrel{a.s.}{\rightarrow}Y_\infty$ (Lemma \ref{lem:distribution-to-probability}).
    Since $g\geq 0$ be continuous, $g(Y_n)\stackrel{a.s.}{\rightarrow}g(Y_\infty)$ and $g(Y_n)\geq 0$ (Theorem \ref{thm:continuous-mapping-theorem}), and the Fatou's Lemma (\ref{thm:fatou-lemma}) implies,
    \begin{equation*}
        \begin{aligned}
            \liminf_{n\rightarrow\infty}Eg(X_n)=&\liminf_{n\rightarrow\infty}Eg(Y_n)\geq E\left(\liminf_{n\rightarrow\infty}g(Y_n)\right)\\
            =&Eg(Y_\infty)=Eg(X_\infty).
        \end{aligned}
    \end{equation*}
\end{solution}

\begin{exercise}
    Suppose $g,h$ are continuous with $g(x)>0$, and $|h(x)|/g(x)\rightarrow 0$ as $|x|\rightarrow\infty$. If $F_{n}\stackrel{d}{\rightarrow}F$ and $\int g(x)\mathrm{d}F_{n}(x)\leq C<\infty,$ then
    \begin{equation*}
        \int h(x)\mathrm{d}F_{n}(x) \rightarrow \int h(x)\mathrm{d}F(x).
    \end{equation*}
\end{exercise}

\begin{solution}
    \begin{equation*}
        \begin{aligned}
            \left|\int h(x)\mathrm{d}F_{n}(x)-\int h(x)\mathrm{d}F(x)\right| =& \left|{\int_{x\in[-M,M]}h(x)\mathrm{d}F_{n}(x)+\int_{x\notin[-M,M]}h(x)\mathrm{d}F_{n}(x)}\right. \\
            & \left.{-\int_{x\in[-M,M]}h(x)\mathrm{d}F(x)-\int_{x\notin[-M,M]}h(x)\mathrm{d}F(x)}\right| \\
            \leq& \left|\int_{x\in[-M,M]}h(x)\mathrm{d}F_{n}(x)-\int_{x\in[-M,M]}h(x)\mathrm{d}F(x)\right| \\
            & + \left|\int_{x\notin[-M,M]}h(x)\mathrm{d}F_{n}(x)-\int_{x\notin[-M,M]}h(x)\mathrm{d}F(x)\right|
        \end{aligned}.
    \end{equation*}

    Let $X_n,1\leq n<\infty$, with distribution $F_n$, so that $X_n\stackrel{a.s.}{\rightarrow}X$ (Lemma \ref{lem:distribution-to-probability}).
    \begin{equation*}
        \left|\int_{x\in[-M,M]}h(x)\mathrm{d}F_{n}(x)-\int_{x\in[-M,M]}h(x)\mathrm{d}F(x)\right| = \left|E\left(h(X_n)-h(X)\right)I_{x\in[-M,M]}\right|.
    \end{equation*}

    By Continuity Mapping Theorem (\ref{thm:continuous-mapping-theorem}), $\lim_{n\rightarrow\infty}\left|E\left(h(X_n)-h(X)\right)I_{x\in[-M,M]}\right| = 0$.

    Since
    \begin{equation*}
        h(x)I_{x\notin[-M,M]}\leq g(x)\sup_{x\notin[-M,M]}\frac{h(x)}{g(x)},
    \end{equation*}
    and by Exercise \ref{ex:fatou-lemma-distribution}
    \begin{equation*}
        Eg(X) \leq \liminf_{n\rightarrow\infty}Eg(X_n)=\liminf_{n\rightarrow\infty}\int g(x)\mathrm{d}F_{n}(x)\leq C<\infty,
    \end{equation*}
    \begin{equation*}
        \begin{aligned}
            \left|\int_{x\notin[-M,M]}h(x)\mathrm{d}F_{n}(x)-\int_{x\notin[-M,M]}h(x)\mathrm{d}F(x)\right| = \left|E\left(h(X_n)-h(X)\right)I_{x\notin[-M,M]}\right| \\
            \leq 2E\max(h(Xn),h(X))I_{x\notin[-M,M]}\leq 2C\sup_{x\notin[-M,M]}\frac{h(x)}{g(x)}. \\
        \end{aligned}
    \end{equation*}

    Hence, let $M\rightarrow\infty$,
    \begin{equation*}
        \lim_{n\rightarrow\infty}\left|\int h(x)\mathrm{d}F_{n}(x)-\int h(x)\mathrm{d}F(x)\right| \leq 2C\sup_{x\notin[-M,M]}\frac{h(x)}{g(x)}\rightarrow 0,
    \end{equation*}
    which means,
    \begin{equation*}
        \int h(x)\mathrm{d}F_{n}(x) \rightarrow \int h(x)\mathrm{d}F(x).
    \end{equation*}
\end{solution}

\begin{exercise}
    Let $X_{1},X_{2},\ldots$ be i.i.d. with $EX_{i}=0$ and $EX_{i}^{2}=\sigma^{2}\in(0,\infty)$. Then
    \begin{equation*}
        \sum_{m=1}^{n}X_{m}/\left(\sum_{m=1}^{n}X_{m}^{2}\right)^{1/2}\stackrel{d}{\rightarrow}\chi.
    \end{equation*}
\end{exercise}

\begin{exercise}
    Show that if $\left|X_{i}\right|\leq M$ and $\sum_{n}\text{Var}\left(X_{n}\right)=\infty$, then
    \begin{equation*}
        \left(S_{n}-E S_{n}\right)/\sqrt{\text{Var}\left(S_{n}\right)}\stackrel{d}{\rightarrow}\chi.
    \end{equation*}
\end{exercise}

\begin{exercise}
    Suppose $EX_{i}=0,EX_{i}^{2}=1$ and $E\left|X_{i}\right|^{2+\delta}\leq C$ for some $0<\delta,C<\infty$. Show that
    \begin{equation*}
        S_{n}/\sqrt{n}\stackrel{d}{\rightarrow}\chi.
    \end{equation*}
\end{exercise}