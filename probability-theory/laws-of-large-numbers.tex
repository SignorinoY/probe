\chapter{Law of Large Numbers}

\section{Weak Law of Large Numbers}

\begin{theorem}[Weak Law of Large Numbers with Finite Variances]
	Let $X_1,X_2,\ldots$ be i.i.d. random variables with $EX_i=\mu$ and $\text{Var}(X_i)\leq C<\infty$. Suppose $S_n=X_1+X_2+\ldots+X_n$, then
	\begin{equation}
		S_n/n\stackrel{L^2}{\rightarrow}\mu,\quad S_n/n\stackrel{p}{\rightarrow}\mu.
	\end{equation}
\end{theorem}

\begin{proof}

\end{proof}

\begin{theorem}[Weak Law of Large Numbers without i.i.d.]
	Let $X_1,X_2,\ldots$ be random variables, Suppose $S_n=X_1+X_2+\ldots+X_n$, $\mu_n=ES_n$, $\sigma_n^2=\text{Var}(S_n)$, if $\sigma_n^2/b_n^2\rightarrow 0$, then
	\begin{equation}
		\frac{S_n-\mu_n}{b_n}\stackrel{p}{\rightarrow}0.
	\end{equation}
\end{theorem}

\begin{proof}

\end{proof}

\begin{theorem}[Weak Law of Large Numbers for Triangular Arrays]
	For each $n$, let $X_{n,m},1\leq m\leq n$, be independent random variables. Suppose $b_n>0$ with $b_n\rightarrow\infty$, $\bar{X}_{n,m}=X_{n,m}I_{\left(X_{n,m}\leq b_n\right)}$, if
	\begin{enumerate}
		\item $\sum_{m=1}^{n}P\left(\left|X_{n,m}\right|>b_{n}\right)\rightarrow 0$, and
		\item $b_{n}^{-2}\sum_{m=1}^{n}E\bar{X}_{n,m}^{2}\rightarrow 0$.
	\end{enumerate}
	Suppose $S_{n}=X_{n, 1}+\cdots+X_{n,n}$ and $a_{n}=\sum_{m=1}^{n}E\bar{X}_{n,m}$, then
	\begin{equation}
		\frac{S_n-a_n}{b_n}\stackrel{p}{\rightarrow}0.
	\end{equation}
\end{theorem}

\begin{proof}

\end{proof}

\begin{theorem}[Weak Law of Large Numbers by Feller]
	Let $X_1,X_2,\ldots$ be i.i.d. random variables with
	\begin{equation}
		\lim_{x\rightarrow 0}xP(|X_i|>x)=0.
	\end{equation}
	Suppose $S_n=X_1+X_2+\ldots+X_n$, $\mu_n=E\left(X_1I_{(|X_1|<n)}\right)$, then
	\begin{equation}
		S_n/n-\mu_n\stackrel{p}{\rightarrow}0.
	\end{equation}
\end{theorem}

\begin{proof}

\end{proof}

\begin{theorem}[Weak Law of Large Numbers] \label{thm:WLLN}
	Let $X_1,X_2,\ldots$ be i.i.d. random variables with $E|X_i|<\infty$. Suppose $S_n=X_1+X_2+\ldots+X_n$, $\mu=EX_i$, then
	\begin{equation}
		S_n/n\stackrel{p}{\rightarrow}\mu.
	\end{equation}
\end{theorem}

\begin{proof}

\end{proof}

\begin{remark}
	$E|X_i|=\infty$
\end{remark}

\section{Strong Law of Large Numbers}

\subsection{Borel-Cantelli Lemmas}

\begin{lemma}[Borel-Cantelli Lemma] \label{lem:borel-cantelli-lemma}
	If $\sum_{n=1}^{\infty}P\left(A_{n}\right)<\infty$, then
	\begin{equation}
		P\left(A_{n}\text{ i.o. }\right)=0.
	\end{equation}
\end{lemma}

\begin{lemma}[The Second Borel-Cantelli Lemma]
	If $\{A_n\}$ are independent with $\sum_{n=1}^{\infty}P\left(A_{n}\right)=\infty$, then,
	\begin{equation}
		P\left(A_{n}\text{ i.o. }\right)=1.
	\end{equation}
\end{lemma}

\begin{corollary}
	Suppose $\{A_{n}\}$ are independent with $P\left(A_{n}\right)<1,\forall n$. If $P\left(\cup_{n=1}^{\infty}A_{n}\right)=1$ then
	\begin{equation}
		\sum_{n=1}^{\infty}P\left(A_{n}\right)=\infty,
	\end{equation}
	and hence $P\left(A_{n}\text{ i.o. }\right)=1$
\end{corollary}

\begin{proof}

\end{proof}

\subsection{Strong Law of Large Numbers}

\begin{theorem}[Strong Law of Large Numbers] \label{thm:SLLN}
	Let $X_1,X_2,\ldots$ be i.i.d. random variables with $E|X_i|<\infty$. Suppose $S_n=X_1+X_2+\ldots+X_n$, $\mu=EX_i$, then
	\begin{equation}
		S_n/n\stackrel{a.s.}{\rightarrow}\mu.
	\end{equation}
\end{theorem}

\section{Uniform Law of Large Numbers}

\begin{theorem}[Uniform Law of Large Numbers] \label{thm:ULLN}
	Suppose
	\begin{enumerate}
		\item $\Theta$ is compact.
		\item $g\left(X_{i},\theta\right)$ is continuous at each $\theta\in\Theta$ almost sure.
		\item $g\left(X_{i},\theta\right)$ is dominated by a function $G\left(X_{i}\right)$, i.e. $\left|g\left(X_{i},\theta\right)\right|\leq G\left(X_{i}\right)$.
		\item $EG\left(X_{i}\right)<\infty$.
	\end{enumerate}
	Then
	\begin{equation}
		\sup_{\theta\in\Theta}\left|n^{-1}\sum_{i=1}^{n}g\left(X_{i},\theta\right)-Eg\left(X_{i},\theta\right)\right|\stackrel{p}{\rightarrow}0.
	\end{equation}
\end{theorem}

\begin{proof}
	Suppose
	\begin{equation*}
		\Delta_{\delta}\left(X_{i},\theta_{0}\right)=\sup_{\theta\in B\left(\theta_{0},\delta\right)}g\left(X_{i},\theta\right)-\inf_{\theta\in B\left(\theta_{0},\delta\right)}g\left(X_{i},\theta\right).
	\end{equation*}

	Since (i) $\Delta_{\delta}\left(X_{i},\theta_{0}\right)\stackrel{a.s.}{\rightarrow} 0$ by condition (2), (ii) $\Delta_{\delta}\left(X_{i},\theta_{0}\right) \leq 2\sup_{\theta\in\Theta}\left|g\left(X_{i},\theta\right)\right|\leq 2G\left(X_{i}\right)$ by condition (3) and (4). Then
	\begin{equation*}
		E\Delta_{\delta}\left(X_{i},\theta_{0}\right)\rightarrow 0,\text{ as }\delta\rightarrow 0.
	\end{equation*}

	So, for all $\theta\in\Theta$ and $\varepsilon>0$, there exists $\delta_{\varepsilon}(\theta)$ such that
	\begin{equation*}
		E\left[\Delta_{\delta_{\varepsilon}(\theta)}\left(X_{i},\theta\right)\right]<\varepsilon.
	\end{equation*}

	Since $\Theta$ is compact, we can find a finite subcover, such that $\Theta$ is covered by
	\begin{equation*}
		\cup_{k=1}^{K}B\left(\theta_{k}, \delta_{\varepsilon}\left(\theta_{k}\right)\right).
	\end{equation*}

	\begin{equation*}
		\begin{aligned}
			     & \sup_{\theta\in\Theta}\left[n^{-1}\sum_{i=1}^{n}g\left(X_{i},\theta\right)-Eg\left(X_{i},\theta\right)\right]                                                                                                                                                             \\
			=    & \max_{k}\sup_{\theta\in B\left(\theta_{k},\delta_{\varepsilon}\left(\theta_{k}\right)\right)}\left[n^{-1}\sum_{i=1}^{n}g\left(X_{i},\theta\right)-Eg\left(X_{i},\theta\right)\right]                                                                                      \\
			\leq & \max_{k}\left[n^{-1}\sum_{i=1}^{n}\sup_{\theta\in B\left(\theta_{k},\delta_{\varepsilon}\left(\theta_{k}\right)\right)}g\left(X_{i},\theta\right)-E\inf_{\theta\in B\left(\theta_{k},\delta_{\varepsilon}\left(\theta_{k}\right)\right)}g\left(X_{i},\theta\right)\right] \\
		\end{aligned}.
	\end{equation*}

	Since
	\begin{equation*}
		E\left|\sup_{\theta\in B\left(\theta_{k},\delta_{c}\left(\theta_{k}\right)\right)}g\left(X_{i},\theta\right)\right|\leq EG\left(X_{i}\right)<\infty,
	\end{equation*}
	by the Weak Law of Large Numbers (Theorem \ref{thm:WLLN}),

	\begin{equation*}
		\begin{aligned}
			=    & o_{p}(1)+\max_{k}\left[E\sup_{\theta\in B\left(\theta_{k},\delta_{\varepsilon}\left(\theta_{k}\right)\right)}g\left(X_{i},\theta\right)-E\inf_{\theta\in B\left(\theta_{k},\delta_{\varepsilon}\left(\theta_{k}\right)\right)}g\left(X_{i},\theta\right)\right] \\
			=    & o_{p}(1)+\max_{k}E\Delta_{\delta_{\varepsilon}\left(\theta_{k}\right)}\left(X_{i},\theta_{k}\right)                                                                                                                                                             \\
			\leq & o_{p}(1)+\varepsilon
		\end{aligned}
	\end{equation*}

	By analogous argument,
	\begin{equation*}
		\inf_{\theta\in\Theta}\left[n^{-1}\sum_{i=1}^{n}g\left(X_{i},\theta\right)-Eg\left(X_{i},\theta\right)\right]\geq o_{p}(1)-\varepsilon.
	\end{equation*}

	The desired result follows from the above equation by the fact that $\varepsilon$ is chosen arbitrarily.
\end{proof}
