\chapter{Random Variables}

% \begin{introduction}
%     \item Probability Space
%     \item Random Variables
%     \item Distributions
%     \item Expected Value
%     \item Independence
%     \item Characteristic Functions
% \end{introduction}

\section{Probability Space}

\begin{definition}[Probability Space]
	A probability space is a triple $(\Omega,\mcF,P)$ consisting of:
	\begin{enumerate}
		\item the sample space $\Omega$: an arbitrary non-empty set.
		\item the $\sigma$-algebra $\mcF\subseteq 2^{\Omega}$: a set of subsets of $\Omega$, called events.
		\item the probability measure $P:\mcF \rightarrow[0,1]$: a function on $\mcF$ which is a measure function.
	\end{enumerate}
\end{definition}

\section{Random Variables}

\begin{definition}[Random Variable]
	A random variable is a measurable function $X:\Omega\rightarrow S$ from a set of possible outcomes $(\Omega,\mcF)$ to a measurable space $(S,\mathcal{S})$, that is,
	\begin{equation}
		X^{-1}(B)\equiv\{\omega:X(\omega)\in B\}\in\mcF\quad \forall B\in\mathcal{S}.
	\end{equation}
	Typically, $(S,\mathcal{S})=(R^d,\mathcal{R}^d)\quad(d>1)$.
\end{definition}

How to prove that functions are measurable?

\begin{theorem}
	If $\{\omega:X(\omega)\in A\}\in\mcF$ for all $A\in\mathcal{A}$ and $\mathcal{A}$ generates $\mathcal{S}$, then $X$ is measurable.
\end{theorem}

\begin{enumerate}
	\item
\end{enumerate}

\section{Distributions}

\subsection{Definition of Distributions}

\begin{definition}[Distribution]
	A distribution of random variable $X$ is a probability function $P:\mathcal{R}\rightarrow\bbR$ by setting
	\begin{equation}
		\mu(A)=P(X\in A)=P\left(X^{-1}(A)\right),\quad\text{for}A\in\mathcal{R}.
	\end{equation}
\end{definition}

\begin{definition}[Distribution Function]
	The distribution of a random variable $X$ is usually described by giving its \textbf{distribution function},
	\begin{equation}
		F(x)=P(X\leq x).
	\end{equation}
\end{definition}

\begin{definition}[Density Function]
	If the distribution function $F(x)=P(X\leq x)$ has the form
	\begin{equation*}
		F(x)=\int_{-\infty}^{x}f(y)\dif y,
	\end{equation*}
	that $X$ has density function $f$.
\end{definition}

\subsection{Properties of Distributions}

\begin{theorem}[Properties of Distribution Function] \label{thm:distribution-function-property}
	Any distribution function $F$ has the following properties,
	\begin{enumerate}
		\item $F$ is nondecreasing.
		\item $\lim_{x\rightarrow\infty}F(x)=1,\lim_{x \rightarrow-\infty}F(x)=0$.
		\item $F$ is right continuous, i.e., $\lim_{y \downarrow x} F(y)=F(x)$.
		\item If $F(x-)=\lim_{y\uparrow x}F(y)$, then $F(x-)=P(X<x)$.
		\item $P(X=x)=F(x)-F(x-)$.
	\end{enumerate}
\end{theorem}

\begin{proof}

\end{proof}

\begin{theorem}
	If $F$ satisfies (1), (2), and (3) in Theorem \ref{thm:distribution-function-property}, then it is the distribution function of some random variable.
\end{theorem}

\begin{proof}

\end{proof}

\begin{theorem}
	A distribution function has at most countably many discontinuities
\end{theorem}

\begin{proof}

\end{proof}

\subsection{Families of Distributions}

\subsubsection{Exponential Family}

\begin{definition}[Exponential Family] \label{def:exponential-family}
	An exponential family of probability distributions is those distributions whose density is defined to be
	\begin{equation}
		f\left(y\mid\theta,\phi\right)=\exp\left[\frac{y\theta-b(\theta)}{a(\phi)}+c(y,\phi)\right]
	\end{equation}
\end{definition}

\begin{property}
	The exponential family has the following properties,
	\begin{equation*}
		E(Y)=b^{\prime}(\theta)\quad\operatorname{Var}(Y)=b^{\prime\prime}(\theta)a(\phi).
	\end{equation*}
\end{property}

\begin{proof}

\end{proof}

\begin{landscape}
	\begin{table}[hpt]
		\centering
		\caption{Common Distributions of Exponential Family}
		\begin{tabular}{ccccccccc}
			\toprule
			Distribution & Parameter(s)           & $\theta$                                & $\phi$       & $b(\theta)$                     & $a(\phi)$ & $c(y,\phi)$                                                   & $E(Y)$                            & $\operatorname{Var}(Y)$                            \\
			\midrule
			Normal       & $N(\mu,\sigma^2)$      & $\mu$                                   & $\sigma^{2}$ & $\frac{\theta^{2}}{2}$          & $\phi$    & $-\frac{1}{2}\left[\frac{y^{2}}{\phi}+\log (2\pi\phi)\right]$ & $\theta$                          & $\phi$                                             \\
			Bernoulli    & $\text{Bern}(p)$       & $\log\left(\frac{p}{1-p}\right)$        & $1$          & $\log\left(1+e^{\theta}\right)$ & $1$       & $0$                                                           & $\frac{e^{\theta}}{1+e^{\theta}}$ & $\frac{e^{\theta}}{\left(1+e^{\theta}\right)^{2}}$ \\
			Poisson      & $P(\mu)$               & $\log(\mu)$                             & $1$          & $\mathrm{e}^{\theta}$           & $1$       & $-\log(y!)$                                                   & $\mathrm{e}^{\theta}$             & $\mathrm{e}^{\theta}$                              \\
			Gamma        & $\Gamma(\alpha,\beta)$ & $\log\left(\frac{\alpha}{\beta}\right)$ & $1$          & $-\log(-\theta)$                & $1$       & $-\log\left(\Gamma(\alpha)\right)+(\alpha-1)\log(y)-y$        & $\frac{\alpha}{\beta}$            & $\frac{\alpha}{\beta^{2}}$                         \\
			\bottomrule
		\end{tabular}
	\end{table}
\end{landscape}

\section{Expected Value}

\begin{definition}[Expectation] \label{def:expectation}

\end{definition}

\begin{theorem}[Bounded Convergence theorem] \label{thm:bounded-convergence-theorem}

\end{theorem}

\begin{theorem}[Fatou's Lemma] \label{thm:fatou-lemma}
	If $X_n \geq 0$, then
	\begin{equation}
		\liminf _{n \rightarrow \infty} E X_{n} \geq E\left(\liminf _{n \rightarrow \infty} X_{n}\right).
	\end{equation}
\end{theorem}

\begin{theorem}[Monotone Convergence theorem] \label{thm:monotone-convergence}
	If $0 \leq X_{n} \uparrow X$, then
	\begin{equation}
		E X_{n} \uparrow E X.
	\end{equation}
\end{theorem}

\begin{theorem}[Dominated Convergence theorem] \label{thm:dominated-convergence}
	If $X_{n} \rightarrow X$ a.s., $\left|X_{n}\right| \leq Y$ for all $n$, and $E Y<\infty$, then
	\begin{equation}
		E X_{n} \rightarrow E X.
	\end{equation}
\end{theorem}

\section{Independence}

\subsection{Definition of Independence}

\begin{definition}[Independence]
	\begin{enumerate}
		\item Two events $A$ and $B$ are independent if $P(A \cap B)=P(A) P(B)$.
		\item Two random variables $X$ and $Y$ are independent if for all $C,D\in\mathcal{R}$
		      \begin{equation}
			      P(X\in C,Y\in D)=P(X\in C)P(Y\in D).
		      \end{equation}
		\item Two $\sigma$-fields $\mcF$ and $\mathcal{G}$ are independent if for all $A\in\mcF$ and $B\in\mathcal{G}$ the events $A$ and $B$ are independent.
	\end{enumerate}
\end{definition}

The second definition is a special case of the third.

\begin{theorem}
	\begin{enumerate}
		\item If $X$ and $Y$ are independent then $\sigma(X)$ and $\sigma(Y)$ are independent.
		\item Conversely, if $\mcF$ and $\mathcal{G}$ are independent, $X\in\mcF$ and $Y\in\mathcal{G}$, then $X$ and $Y$ are independent.
	\end{enumerate}
\end{theorem}

The first definition is, in turn, a special case of the second.

\begin{theorem}
	\begin{enumerate}
		\item If $A$ and $B$ are independent, then so are $A^{c}$ and $B, A$ and $B^{c}$, and $A^{c}$ and $B^{c}$.
		\item Conversely, events $A$ and $B$ are independent if and only if their indicator random variables $1_{A}$ and $1_{B}$ are independent.
	\end{enumerate}
\end{theorem}

The definition of independence can be extended to the infinite collection.

\begin{definition}
	An infinite collection of objects ($\sigma$-fields, random variables, or sets) is said to be independent if every finite subcollection is,
	\begin{enumerate}
		\item $\sigma$-fields $\mcF_{1},\mcF_{2},\ldots,\mcF_{n}$ are independent if whenever $A_{i}\in\mcF_{i}$ for $i=1, \ldots,n$, we have
		      \begin{equation}
			      P\left(\cap_{i=1}^{n}A_{i}\right)=\prod_{i=1}^{n}P\left(A_{i}\right).
		      \end{equation}
		\item Random variables $X_{1},\ldots,X_{n}$ are independent if whenever $B_{i}\in\mathcal{R}$ for $i=1,\ldots,n$ we have
		      \begin{equation}
			      P\left(\cap_{i=1}^{n}\left\{X_{i}\in B_{i}\right\}\right)=\prod_{i=1}^{n}P\left(X_{i}\in B_{i}\right).
		      \end{equation}
		\item Sets $A_{1},\ldots,A_{n}$ are independent if whenever $I\subset\{1,\ldots,n\}$ we have
		      \begin{equation}
			      P\left(\cap_{i\in I}A_{i}\right)=\prod_{i\in I}P\left(A_{i}\right).
		      \end{equation}
	\end{enumerate}
\end{definition}

\subsection{Sufficient Conditions for Independence}

\subsection{Independence, Distribution, and Expectation}

\begin{theorem}
	Suppose $X_{1},\ldots,X_{n}$ are independent random variables and $X_{i}$ has distribution $\mu_{i}$, then $\left(X_{1},\ldots,X_{n}\right)$ has distribution $\mu_{1}\times\cdots\times\mu_{n}$.
\end{theorem}

\begin{theorem}
	If $X_{1},\ldots,X_{n}$ are independent and have
	\begin{enumerate}
		\item $X_{i} \geq 0$ for all i, or
		\item $E\left|X_{i}\right|<\infty$ for all $i$.
	\end{enumerate}
	then
	\begin{equation}
		E\left(\prod_{i=1}^{n}X_{i}\right)=\prod_{i=1}^{n}EX_{i}
	\end{equation}
\end{theorem}

\subsection{Sums of Independent Random Variables}

\begin{theorem}[Convolution for Random Variables]
	\begin{enumerate}
		\item If $X$ and $Y$ are independent, $F(x)=P(X\leq x)$, and $G(y)=P(Y\leq y)$, then
		      \begin{equation}
			      P(X+Y\leq z)=\int F(z-y)\dif G(y).
		      \end{equation}
		\item If $X$ and $Y$ are independent,  $X$ with density $f$ and $Y$ with distribution function $G$, then $X+Y$ has density
		      \begin{equation}
			      h(x)=\int f(x-y)\dif G(y).
		      \end{equation}
		      Suppose $Y$ has density $g$, the last formula can be written as
		      \begin{equation}
			      h(x)=\int f(x-y)g(y)\dif y.
		      \end{equation}
		\item If $X$ and $Y$ are independent, integral-valued random variables, then
		      \begin{equation}
			      P(X+Y=n)=\sum_{m}P(X=m)P(Y=n-m).
		      \end{equation}
	\end{enumerate}
\end{theorem}

\section{Moments}

\begin{lemma}
	If $Y>0$ and $p>0$, then
	\begin{equation}
		E(Y^p)=\int_{0}^{\infty}py^{p-1}P(Y>y)\dif y.
	\end{equation}
\end{lemma}

\section{Characteristic Functions}

\subsection{Definition of Characteristic Functions}

\begin{definition}[Characteristic Function] \label{def:characteristic-function}
	If $X$ is a random variable, we define its characteristic function (ch. f) by
	\begin{equation}
		\varphi(t)=E\left(e^{itX}\right)=E\left(\cos tX\right)+i E\left(\sin tX\right).
	\end{equation}
\end{definition}

\begin{remark}
	Euler Equation.
\end{remark}

\subsection{Properties of Characteristic Functions}

\begin{theorem}[Properties of Characteristic Function] \label{thm:characteristic-function-property}
	Any characteristic function has the following properties:
	\begin{enumerate}
		\item $\varphi(0) = 1$,
		\item $\varphi(-t) = \overline{\varphi(t)}$,
		\item $|\varphi(t)| =|Ee^{itX}| \leq E|e^{itX}| = 1$,
		\item $\varphi(t)$ is uniformly continuous on $(-\infty,\infty)$,
		\item $Ee^{it(aX+b)}=e^{itb}\varphi(at)$,
		\item  If $X_1$ and $X_2$ are independent and have ch.f.'s $\varphi_1$ and $\varphi_2$, then $X_1+X_2$ has ch.f. $\varphi_1(t)\varphi_2(t)$.
	\end{enumerate}
\end{theorem}

\begin{proof}

\end{proof}

\subsection{The Inversion Formula}

The characteristic function uniquely determines the distribution. This and more is provided by:
\begin{theorem}[The Inversion Formula]
	Let $\varphi(t)=\int e^{itx}\mu(\dif x)$ where $\mu$ is a probability measure. If $a<b$, then
	\begin{equation}
		\lim _{T \rightarrow \infty}(2 \pi)^{-1} \int_{-T}^{T} \frac{e^{-i t a}-e^{-i t b}}{i t} \varphi(t) d t=\mu(a, b)+\frac{1}{2} \mu(\{a, b\})
	\end{equation}
\end{theorem}

\begin{proof}

\end{proof}

\begin{theorem}
	If $\int|\varphi(t)|\dif t<\infty$, then $\mu$ has bounded continuous density
	\begin{equation}
		f(y)=\frac{1}{2\pi}\int e^{-ity}\varphi(t) \dif t.
	\end{equation}
\end{theorem}

\begin{proof}

\end{proof}

\subsection{Moments and Derivatives}

\begin{theorem}
	If $\int|x|^{n}\mu(d x)<\infty,$ then its characteristic function $\varphi$ has a continuous derivative of order $n$ given by
	\begin{equation}
		\varphi^{(n)}(t)=\int(i x)^{n}e^{itx}\mu(\dif x).
	\end{equation}
\end{theorem}

\begin{theorem}
	If $E|X|^{2}<\infty$ then
	\begin{equation}
		\varphi(t)=1+itEX-t^{2}E\left(X^{2}\right)/2+o\left(t^{2}\right).
	\end{equation}
\end{theorem}

\begin{theorem}
	If $\limsup_{h\downarrow 0}\{\varphi(h)-2\varphi(0)+\varphi(-h)\}/h^{2}>-\infty$, then
	\begin{equation}
		E|X|^{2}<\infty.
	\end{equation}
\end{theorem}
