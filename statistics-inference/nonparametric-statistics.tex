\chapter{Nonparametric Statistics}

\section{Probability Distribution}

\subsection{Cumulative Distribution Function}

Let $X_{1},\ldots,X_{n}\sim F$ where $F(x)=\mathbb{P}(X\leq x)$ is a distribution function on the real line.

\begin{definition}[Empirical Cumulative Distribution Function]
	The empirical cumulative distribution function $\widehat{F}_{n}$ is the CDF that puts mass $1/n$ at each data point $X_{i}$, that,
	\begin{equation}
		\widehat{F}_{n}(x)=\frac{1}{n}\sum_{i=1}^{n}I\left(X_{i}\leq x\right)
	\end{equation}
\end{definition}

Denote $Y=\sum_{i=1}^{n}I\left(X_{i}\leq x\right)$, then $Y$ is a binomial random variable with parameters $n$ and $F(x)$, that is, $Y\sim b(n,F(x))$. Therefore, we have
\begin{equation*}
	E\left[\hat{F}_{n}(x)\right]=F(x),\quad\Var\left[\hat{F}_{n}(x)\right]=\frac{1}{n}F(x)[1-F(x)]
\end{equation*}
so that $\hat{F}_{n}(a)$ is unbiased and its variance is of order $1/n$. In addition, it follows from CLT \eqref{thm:classic-central-limit-theorem} that
\begin{equation*}
	\sqrt{n}\left[\hat{F}_{n}(x)-F(x)\right]\stackrel{d}{\rightarrow}\mcN(0,F(x)[1-F(x)]).
\end{equation*}
Thus, $\hat{F}_{n}(x)$ is a consistent estimator of $F(x)$ for each fixed $x$.

However, a much stronger consistency property can be asserted if the difference between $\hat{F}_{n}(x)$ and $F(x)$ is considered not only for a fixed $x$ but simultaneously for all $x$, namely
\begin{equation*}
	D_{n}=\sup _x\left|\hat{F}_{n}(x)-F(x)\right| \xrightarrow{P} 0,\quad n \rightarrow \infty.
\end{equation*}
For a still stronger result, see, for example, Serfling (1980, Section 2.1.4) or Billingsley (1986, Theorem 20.6).

\subsection{Probability Density Function}

\paragraph{Histogram Estimator}
Since
\begin{equation*}
	f(x)=\lim_{h\rightarrow0}\frac{F(x+h)-F(x-h)}{2h},
\end{equation*}
one might consider the estimator
\begin{equation*}
	\hat{f}_{n}(x)=\frac{\hat{F}_{n}(x+h)-\hat{F}_{n}(x-h)}{2h}.
\end{equation*}
and hope that with $h=h_{n}$ tending to 0 as $n\rightarrow\infty$, the estimator $\hat{f}_{n}(x)$, the so-called Rosenblatt estimator, will be consistent for $f(x)$.

\begin{proof}
	The basic properties of $\hat{p}_{n}$ are obtained from the fact that
	\begin{equation*}
		n\left[\hat{F}_{n}(x+h_{n})-\hat{F}_{n}(x-h_{n})\right]\sim b(n,p),
	\end{equation*}
	where $p=F(x+h_{n})-F(x-h_{n})$. It follows that
	\begin{equation*}
		\bbE\left[\hat{f}_{n}(x)\right]=\frac{F(x+h_{n})-F(x-h_{n})}{2h}.
	\end{equation*}
	The bias is therefore
	\begin{equation*}
		b(x)=\frac{F(x+h_{n})-F(x-h_{n})}{2h}-\lim_{h_{n}\rightarrow0}\frac{F(x+h_{n})-F(x-h_{n})}{2h}
	\end{equation*}
	which tends to be zero, provided $h_{n}\rightarrow0$, as $n\rightarrow\infty$.

	Similarly, the variance of $\hat{f}_{n}(x)$ is
	\begin{equation*}
		\Var\left[\hat{f}_{n}(x)\right]=\frac{p(1-p)}{4nh^{2}}.
	\end{equation*}
	As $h_{n} \rightarrow 0$, the value of $p$
	\begin{equation*}
		p_{n}=F\left(x+h_{n}\right)-F\left(x-h_{n}\right) \rightarrow 0.
	\end{equation*}
	thus,
	\begin{equation*}
		\Var\left[\hat{f}_{n}(x)\right]\sim\frac{p_{n}}{2h_{n}}\cdot\frac{1}{2nh_{n}}.
	\end{equation*}
	Since the first factor on the right side tends to $f(x)>0$, $\Var\left[\hat{f}_{n}(x)\right] \rightarrow 0$ as $h_{n}\rightarrow 0$ if in addition $nh_{n}\rightarrow\infty$, that is, if $h_{n}$ tends to 0 more slowly than $1/n$ or, equivalently, if $\frac{1}{n}=o\left(h_{n}\right)$.
	From these results, we immediately obtain sufficient conditions for the consistency of $\hat{f}_{n}(x)$.
\end{proof}

It is interesting to note that $\hat{f}_{n}(x)$ is itself a probability density. Since it is non-negative, one only needs to show that $\int_{-\infty}^{\infty}\hat{f}_{n}(x)\dif x=1$. This is easily seen by writing
\begin{equation*}
	\hat{f}_{n}(x)=\frac{1}{2nh}\sum_{i=1}^{n}I(x-h_{n}\leq X_{i}\leq x+h_{n})
\end{equation*}
Then
\begin{equation*}
	\int\hat{f}_{n}(x)\dif x=\frac{1}{2nh}\sum_{i=1}^{n}\int_{-\infty}^{\infty}I_i(x)\dif x=\frac{1}{2nh}\sum_{i=1}^{n}\int_{x_j-h}^{x_j+h}\dif x=1.
\end{equation*}

\begin{remark}
	Although $\hat{f}_{n}(x)$ is consistent for estimating $f(x)$ when the $h_{n}$'s satisfy $h_{n}\rightarrow0$ and $nh_{n}\rightarrow\infty$, note that $\hat{f}_{n}(x)$ as an estimator of $f(x)$ involves two approximations: $f(x)$ by $[F(x+h_{n})-F(x-h_{n})]/2h_{n}$ and the latter by $\hat{f}_{n}(x)$. As a result, the estimator turns out to be less accurate than one might have hoped.
	Besides, the estimator $\hat{f}_{n}(x)$, although a density, is a step function with discontinuities at every point $x_{i}\pm h_{n},i=1,\ldots,n$. If we assume the true $f(x)$ to be a smooth density, we may prefer its estimator also to be smooth.
\end{remark}

\paragraph{Kernel Density Estimation}

We can rewrite it in the form:
\begin{equation*}
	\hat{f}_{n}(x)=\frac{1}{2nh_{n}}\sum_{i=1}^{n}I\left(x-h_{n}<X_{i}\leq x+h_{n}\right)=\frac{1}{nh_{n}}\sum_{i=1}^{n}K_{0}\left(\frac{X_i-x}{h}\right),
\end{equation*}
where $K_0(u)=\frac{1}{2}I(-1<u\leq 1)$. A simple generalization of the Rosenblatt estimator is given by
\begin{equation}
	\label{eq:kernel-density-estimator}
	\hat{f}_{n}(x)=\frac{1}{nh_{n}}\sum_{i=1}^{n}K\left(\frac{X_{i}-x}{h_{n}}\right),
\end{equation}
where $K:\bbR\rightarrow\bbR$ is an integrable function satisfying $\int K(u)\dif\mu=1$.
Such a function $K$ is called a kernel and the parameter $h_{n}$ is called a bandwidth of the estimator \eqref{eq:kernel-density-estimator}.
The function $x\mapsto\hat{f}_{n}(x)$ is called the kernel density estimator or the Parzen-Rosenblatt estimator.
In addition, we shall restrict attention to kernel $K$ that are symmetric about $0$.

In generalization of \eqref{eq:kernel-density-estimator}, note that $\hat{f}_{n}$ is a probability density since it is non-negative and since
\begin{equation*}
	\int\hat{f}_{n}(x)\dif x=\frac{1}{n}\sum_{i=1}^{n}\int K\left(\mu-x_i\right)\dif\mu=\frac{1}{n}\sum_{i=1}^{n}\int K(\mu)\dif\mu=1.
\end{equation*}

\begin{theorem}
	Let the density $f$ satisfies $f(x)\leq f_{\max}<\infty$ for all $x$ and let $K$ be a kernel that $int K^2(\mu)\dif\mu<\infty$. Then for any sequence $h_{n},n=1,2,\ldots$,
	\begin{equation*}
		\Var\left[\hat{f}_{n}(x)\right]= \frac{f_{\max}}{nh_{n}}\int K^2(\mu)\dif\mu+o\left(\frac{1}{nh_{n}}\right).
	\end{equation*}
\end{theorem}

\begin{proof}
	Denote
	\begin{equation*}
		\eta_{i}(x)=K\left(\frac{x-X_{i}}{h}\right)-\bbE\left[K\left(\frac{x-X_{i}}{h}\right)\right].
	\end{equation*}
	The random variables $\eta_{i}(x)$ are i.i.d with zero mean and variance
	\begin{equation*}
		\begin{aligned}
			\bbE\left[\eta_{i}^2(x)\right]\leq & \bbE\left[K^2\left(\frac{x-X_{i}}{h}\right)\right]                   \\
			=                                  & \int K^2(\frac{x-z}{h})f(z)\dif z\leq f_{\max}h\int K^2(\mu)\dif\mu.
		\end{aligned}
	\end{equation*}
	Then, we have
	\begin{equation*}
		\Var\left[\hat{f}_{n}(x)\right]=\frac{1}{n^2h^2}\sum_{i=1}^{n}\bbE\left[\eta_{i}^2(x)\right]=\frac{1}{nh^2}\bbE\left[\eta_{1}^2(x)\right]\leq \frac{f_{\max}}{nh} \int K^2(\mu)\dif\mu.
	\end{equation*}
\end{proof}

\begin{theorem}[Bias of Kernel Density Estimator]
	Let $f$ be three times differentiable with a bounded third derivative in a neighborhood of $y$ and let $K$ be a kernel symmetric about $0$, with
	\begin{equation*}
		\int K^2(\mu)\dif\mu<\infty,\quad\int\mu^2K(\mu)\dif\mu=\tau^2<\infty,\quad\int|\mu|^3 K(\mu)\dif\mu<\infty.
	\end{equation*}
	Then for any sequence $h_{n},n=1,2,\ldots$,
	\begin{equation*}
		\Bias\left[\hat{f}_{n}(x)\right]=\frac{1}{2}h_{n}^2f^{\prime\prime}(x)\tau^2+o\left(h_{n}^2\right).
	\end{equation*}
\end{theorem}

\begin{proof}
	Suppressing the subscript $n$, we find for the bias of $\hat{f}_{n}(x)$,
	\begin{equation*}
		\begin{aligned}
			\Bias\left[\hat{f}_{n}(x)\right] & =E\left[\frac{1}{n}\sum_{i=1}^{n}\frac{1}{h}K\left(\frac{x-X_i}{h}\right)\right]-f(x) \\
			                                 & =\int\frac{1}{h}K\left(\frac{x-X}{h}\right)f(X)\dif X-f(x)                            \\
			                                 & =\int K(\mu)[f(x-h\mu)-f(x)]\dif\mu.
		\end{aligned}
	\end{equation*}
	By the Taylor expansion of $f(x-h\mu)$ about $x$, we have
	\begin{equation*}
		f(x-h\mu)=f(x)+h\mu f^{\prime}(x)+\frac{1}{2}(h\mu)^{2}f^{\prime\prime}(x)+\frac{1}{6}(h\mu)^{3}f^{\prime\prime\prime}(\xi),
	\end{equation*}
	where $\xi$ lies between $x$ and $x-h\mu$. Using the fact that $\int\mu K(\mu)\dif\mu=0$, we therefore have
	\begin{equation*}
		\Bias\left[\hat{f}_{n}(x)\right]=\frac{1}{2}h^{2}f^{\prime\prime}(x)\int\mu^{2}K(\mu)\dif\mu+R_{n},
	\end{equation*}
	where, with $\left|f^{\prime \prime \prime}(z)\right| \leq M$,
	\begin{equation*}
		\left|R_{n}\right| \leq \frac{M h^3}{6} \int|z|^3 K(z) d z=o\left(h^2\right),
	\end{equation*}
	which proves the first part of the theorem.
\end{proof}

\section{Kernel Methods}

\subsection{Positive Definite Kernels}

\begin{definition}[Positive Definite Kernel]
	Let $\mcX$ be a set, a function $K:\mcX\times\mcX\rightarrow\bbR$ is called a positive definite kernel on $\mcX$ if and only if it is
	\begin{enumerate}
		\item symmetric, that is,
		      \begin{equation}
			      K\left(\bfx,\bfx^{\prime}\right)=K\left(\bfx^{\prime},\bfx\right),\quad\forall\bfx,\bfx^{\prime}\in\mcX
		      \end{equation}
		\item positive definite, that is,
		      \begin{equation}
			      \sum_{i=1}^{n}\sum_{j=1}^{n}c_{i}c_{j}K\left(\bfx_{i},\bfx_{j}\right)\geq 0,
		      \end{equation}
		      holds for any $x_{1},\ldots,x_{n}\in\mcX$, given $n\in\mathbb{N},c_{1},\ldots,c_{n}\in\bbR$.
	\end{enumerate}
\end{definition}

\subsubsection{Construction of the Reproducing Kernel Hilbert Space}

\begin{theorem}[Morse-Aronszajn's Theorem] \label{thm:morse-aronszajn}
	For any set $\mcX$, suppose $K:\mcX\times\mcX\rightarrow\bbR$ is positive definite, then there is a unique RKHS $\mathcal{H}\subset\bbR^{\mcX}$ with reproducing kernel $K$.
\end{theorem}

\begin{proof}
	\begin{enumerate}
		\item How to build a valid pre-RKHS $\mathcal{H}_{0}$?

		      Consider the vector space $\mathcal{H}_{0}\subset\mathcal{R}^{\mcX}$ spanned by the functions $\left\{K\left(\cdot,\bfx\right)\right\}_{\bfx\in\mcX}$. For any $f,g\in\mathcal{H}_{0}$, suppose
		      \begin{equation*}
			      f=\sum_{i=1}^{m}a_{i}K\left(\cdot,\bfx_{i}\right),\quad g=\sum_{j=1}^{n}b_{j}K\left(\cdot,\mathbf{y}_{j}\right)
		      \end{equation*}
		      and let the inner product of $\mathcal{H}_{0}$ be
		      \begin{equation}
			      \left\langle f,g\right\rangle=\sum_{i=1}^{m}\sum_{j=1}^{n}a_{i}b_{j}K\left(\bfx_{i},\mathbf{y}_{j}\right)
			      \label{eq:definition-pre-rkhs-inner-product}
		      \end{equation}

		      Let $\bfx\in\mcX$,
		      \begin{equation*}
			      \left\langle f,K\left(\cdot,\bfx\right)\right\rangle_{\mathcal{H}_{0}}=\sum_{i=1}^{m}a_{i} K\left(\bfx,\bfx_{i}\right)=f(\bfx)
			      \label{eq:reproducing-kernel}
		      \end{equation*}

		      And, we also have
		      \begin{equation*}
			      \langle f, g\rangle_{\mathcal{H}_{0}}=\sum_{i=1}^{m} a_{i} g\left(\bfx_{i}\right)=\sum_{j=1}^{n} b_{j} f\left(\mathbf{y}_{j}\right)
		      \end{equation*}

		      Suppose
		      \begin{equation*}
			      f=\sum_{i=1}^{m}a_{i}K\left(\cdot,\bfx_{i}\right),\quad g=\sum_{j=1}^{n}b_{j}K\left(\cdot,\mathbf{y}_{j}\right),\quad h=\sum_{k=1}^{p}c_{k}K\left(\cdot,\mathbf{z}_{k}\right)
		      \end{equation*}
		      \begin{enumerate}
			      \item Linearity: For any $\alpha,\beta\in\bbR$, $\left\langle\alpha f+\beta g,h\right\rangle_{\mathcal{H}_{0}}=\alpha\left\langle f,h\right\rangle_{\mathcal{H}_{0}}+\beta\left\langle g,h\right\rangle_{\mathcal{H}_{0}}$.
			            \begin{equation*}
				            \begin{aligned}
					            \left\langle\alpha f+\beta g,h\right\rangle_{\mathcal{H}_{0}}= & \left[\alpha\sum_{i=1}^{m}a_{i}K\left(\cdot,\bfx_{i}\right)+\beta\sum_{j=1}^{n}b_{j}K\left(\cdot,\mathbf{y}_{j}\right)\right]\cdot\sum_{k=1}^{p}c_{k}K\left(\cdot,\mathbf{z}_{k}\right) \\
					            =                                                              & \alpha\sum_{i=1}^{m}\sum_{k=1}^{p}a_{i}c_{k}K\left(\bfx_{i},\mathbf{z}_{k}\right)+\beta\sum_{j=1}^{n}\sum_{k=1}^{p}b_{j}c_{k}K\left(\mathbf{y}_{j},\mathbf{z}_{k}\right)                \\
					            =                                                              & \alpha\left\langle f,h\right\rangle_{\mathcal{H}_{0}}+\beta\left\langle g,h\right\rangle_{\mathcal{H}_{0}}
				            \end{aligned}
			            \end{equation*}
			      \item Conjugate Symmetry: $\langle f,g\rangle_{\mathcal{H}_{0}}=\langle g,f\rangle_{\mathcal{H}_{0}}$.
			            \begin{equation*}
				            \begin{aligned}
					            \langle f,g\rangle_{\mathcal{H}_{0}}= & \sum_{i=1}^{m}\sum_{j=1}^{n}a_{i}b_{j}K\left(\bfx_{i},\mathbf{y}_{j}\right)=\sum_{j=1}^{n}\sum_{i=1}^{m}b_{j}a_{i}K\left(\mathbf{y}_{j},\bfx_{i}\right) \\
					            =                                     & \langle g,f\rangle_{\mathcal{H}_{0}}
				            \end{aligned}
			            \end{equation*}
			      \item Positive Definiteness: $\langle f,f\rangle_{\mathcal{H}_{0}}\geq 0$ and $\langle f,f\rangle_{\mathcal{H}_{0}}=0$ if and only if $f=0$.

			            By positive definiteness of $K$, we have:
			            \begin{equation*}
				            \langle f,f\rangle_{\mathcal{H}_{0}}=\|f\|_{\mathcal{H}_{0}}^{2}=\sum_{i=1}^{m}\sum_{j=1}^{m}a_{i}a_{j}K\left(\bfx_{i},\bfx_{j}\right)\geq 0
			            \end{equation*}

			            As for, $\langle f,f\rangle_{\mathcal{H}_{0}}=0$ if and only if $f=0$, we have,
			            \begin{itemize}
				            \item["$\Rightarrow$"] If $f=0$, that is $f=\sum_{i=1}^{m}a_{i}K\left(\cdot,\bfx_{i}\right)=0$, we have
				                  \begin{equation*}
					                  \langle f,f\rangle_{\mathcal{H}_{0}}=\sum_{i=1}^{m}a_{i}f=0
				                  \end{equation*}
				            \item["$\Leftarrow$"] For $\forall\bfx\in\mcX$, by Cauchy-Schwarz Inequality, we have,
				                  \begin{equation*}
					                  |f(\bfx)|=\left|\left\langle f,K\left(,\cdot{\bfx}\right)\right\rangle_{\mathcal{H}_{0}}\right|\leq\|f\|_{\mathcal{H}_{0}}\cdot K\left(\bfx,\bfx\right)^{\frac{1}{2}}
				                  \end{equation*}
				                  therefore, if $\|f\|_{\mathcal{H}_{0}}=0$, then $f=0$
			            \end{itemize}
		      \end{enumerate}
		      Hence, the definition in equation \eqref{eq:definition-pre-rkhs-inner-product} is a valid inner product, which is a valid pre-RKHS $\mathcal{H}_{0}$.
	\end{enumerate}
\end{proof}

\subsubsection{Examples of Kernels}

\begin{example}[Gaussian Kernel]
	\begin{equation}
		K(\bfx,\mathbf{y})=\exp\left(-\frac{\left\|\bfx-\mathbf{y}\right\|^{2}}{2\sigma^{2}}\right),\quad\bfx,\mathbf{y}\in\bbR^{d}
	\end{equation}
\end{example}

\begin{proof}
	\begin{enumerate}
		\item It is obvious that $K(\bfx,\mathbf{y})$ is symmetric, we only need to show $K(\bfx,\mathbf{y})$ is positive definite.
		      \begin{equation*}
			      \begin{aligned}
				      K(\bfx,\mathbf{y})= & \exp\left(-\frac{\left\|\bfx-\mathbf{y}\right\|^{2}}{2\sigma^{2}}\right)                                                                                                                            \\
				      =                   & \exp\left(-\frac{1}{2\sigma^{2}}\|\bfx\|^{2}\right)\cdot\exp\left(\frac{1}{\sigma^{2}}\left\langle\bfx,\mathbf{y}\right\rangle\right)\cdot\exp\left(-\frac{1}{2\sigma^{2}}\|\mathbf{y}\|^{2}\right)
			      \end{aligned}
		      \end{equation*}

		      By the Taylor expansion of the exponential function, that
		      \begin{equation*}
			      \exp\left(\frac{x}{\sigma^{2}}\right)=\sum_{n=0}^{+\infty}\left\{\frac{x^{n}}{\sigma^{2n}\cdot n!}\right\}
		      \end{equation*}
		      Hence,
		      \begin{equation*}
			      \exp\left(\frac{1}{\sigma^{2}}\left\langle\bfx,\mathbf{y}\right\rangle\right)=\sum_{n=0}^{+\infty}\left\{\frac{\left\langle\bfx,\mathbf{y}\right\rangle^{n}}{\sigma^{2n}\cdot n!}\right\}
		      \end{equation*}

		      By the Multinomial Theorem, we have
		      \begin{equation*}
			      \begin{aligned}
				      \left\langle\bfx,\mathbf{y}\right\rangle^{n}= & \left(\sum_{i=1}^{d}x_{i}y_{i}\right)^{n}=\sum_{k_{1}+k_{2}+\cdots+k_{d}=n}\left[\binom{n}{k_{1},k_{2},\ldots,k_{d}}\prod_{i=1}^{d}\left(x_{i}y_{i}\right)^{k_{i}}\right]                                     \\
				      =                                             & \sum_{k_{1}+k_{2}+\cdots+k_{d}=n}\left[\binom{n}{k_{1},k_{2},\ldots,k_{d}}^{\frac{1}{2}}\prod_{i=1}^{d}x_{i}^{k_{i}}\cdot\binom{n}{k_{1},k_{2},\ldots,k_{d}}^{\frac{1}{2}}\prod_{i=1}^{d}y_{i}^{k_{i}}\right] \\
			      \end{aligned}
		      \end{equation*}
		      Therefore,
		      \begin{equation*}
			      \begin{aligned}
				      K(\bfx,\mathbf{y})= & \exp\left(-\frac{\left\|\bfx-\mathbf{y}\right\|^{2}}{2\sigma^{2}}\right)=\exp\left(-\frac{\|\bfx\|^{2}}{2\sigma^{2}}\right)\cdot\exp\left(-\frac{\|\mathbf{y}\|^{2}}{2\sigma^{2}}\right)\cdot\sum_{n=0}^{+\infty}\left\{\frac{\left\langle\bfx,\mathbf{y}\right\rangle^{n}}{\sigma^{2n}\cdot n!}\right\} \\
				      =                   & \sum_{n=0}^{+\infty}\frac{\exp\left(-\frac{\|\bfx\|^{2}}{2\sigma^{2}}\right)}{\sigma^{n}\cdot\sqrt{n!}}\cdot\frac{\exp\left(-\frac{\|\mathbf{y}\|^{2}}{2\sigma^{2}}\right)}{\sigma^{n}\cdot\sqrt{n!}}\cdot\left\langle\bfx,\mathbf{y}\right\rangle^{n}
			      \end{aligned}
		      \end{equation*}

		      Let
		      \begin{equation*}
			      c_{\sigma,n}\left(\bfx\right)=\frac{\exp\left(-\frac{\|\bfx\|^{2}}{2\sigma^{2}}\right)}{\sigma^{n}\cdot\sqrt{n!}},\quad f_{n,\mathbf{k}}\left(\bfx\right)=\binom{n}{k_{1},k_{2},\ldots,k_{d}}^{\frac{1}{2}}\prod_{i=1}^{d}x_{i}^{k_{i}}
		      \end{equation*}
		      then,
		      \begin{equation*}
			      \begin{aligned}
				      K(\bfx,\mathbf{y})= & \sum_{n=0}^{+\infty}\sum_{k_{1}+k_{2}+\cdots+k_{d}=n}c_{\sigma,n}\left(\bfx\right)f_{n,\mathbf{k}}\left(\bfx\right)\cdot c_{\sigma,n}\left(\mathbf{y}\right)f_{n,\mathbf{k}}\left(\mathbf{y}\right) \\
				      =                   & \left\langle\Phi\left(\bfx\right),\Phi\left(\mathbf{y}\right)\right\rangle                                                                                                                          \\
			      \end{aligned}
		      \end{equation*}
		      where $\Phi\left(\bfx\right)_{\sigma,n,\mathbf{k}}=c_{\sigma,n}\left(\bfx\right)f_{n,\mathbf{k}}\left(\bfx\right)$.

		      \begin{equation*}
			      \begin{aligned}
				      \sum_{i=1}^{n}\sum_{j=1}^{n}c_{i}c_{j}K\left(\bfx_{i},\bfx_{j}\right)= & \sum_{i=1}^{n}\sum_{j=1}^{n}c_{i}c_{j}\left\langle\Phi\left(\bfx_{i}\right),\Phi\left(\bfx_{j}\right)\right\rangle       \\
				      =                                                                      & \left\langle\sum_{i=1}^{n}c_{i}\Phi\left(\bfx_{i}\right),\sum_{i=1}^{n}c_{i}\Phi\left(\bfx_{i}\right)\right\rangle\geq 0
			      \end{aligned}
		      \end{equation*}
		      for any $x_{1},\ldots,x_{n}\in\mcX$, given $n\in\mathbb{N},c_{1},\ldots,c_{n}\in\bbR$, i.e., $K(\bfx,\mathbf{y})$ is positive definite.
	\end{enumerate}
\end{proof}
