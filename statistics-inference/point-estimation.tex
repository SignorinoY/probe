\chapter{Point Estimation}

\section{Maximum Likelihood Estimator}

Suppose that $\bfx_{n}=\left(x_{1},\ldots,x_{n}\right)$, within a parametric family
\begin{equation*}
	p\left(X;\theta_{0}\right)\in\mathcal{P}=\left\{p(X;\bftheta):\bftheta\in\Theta\right\}
\end{equation*}

The maximum likelihood estimate for observed $\bfx_{n}$ is the value $\bftheta\in\Theta$ which maximizes $L_{n}\left(\bftheta\right):=p\left(\bfx_{n};\bftheta\right)$, i.e.,
\begin{equation}
	\hat{\bftheta}=\max_{\bftheta\in\Theta}L_{n}\left(\bftheta\right).
\end{equation}

In practice, it is often convenient to work with the natural logarithm of the likelihood function, called the log-likelihood:
\begin{equation*}
	\ell_{n}\left(\bftheta\right):=\log L_{n}\left(\bftheta\right)
\end{equation*}
Since the logarithm is a monotonic function, the maximum of $\ell_{n}\left(\bftheta\right)$ occurs at the same value of $\bftheta$ as does the maximum of $L_{n}\left(\bftheta\right)$

\subsection{Consistency}

To establish consistency, the following conditions are sufficient:
\begin{enumerate}[label=(C\arabic*)]
	\item Identification: $\bftheta_{0}$ is identified in the sense that if $\bftheta\neq\bftheta_{0}$ and $\bftheta\in\Theta$, then $p(X;\bftheta)\neq p\left(X;\bftheta_{0}\right)$ with respect to the dominating measure $\mu$.
	\item \label{cond:mle-compactness}
	      The parameter space $\Theta$ of the model is compact.
	\item \label{cond:mle-continuity}
	      The function $\log p(X;\bftheta)$ is continuous in $\bftheta$ for almost all values of $x$, i.e.,
	      \begin{equation}
		      P\left[\log p(X;\bftheta)\in C^{0}(\Theta)\right]=1
	      \end{equation}
	\item Dominance: there exists $D(x)$ integrable with respect to the distribution $p\left(X;\bftheta_{0}\right)$ such that $|\log p\left(X;\bftheta\right)|<D(x)$ for all $\bftheta\in\Theta$.
\end{enumerate}

\begin{lemma}
	If $\bftheta_{0}$ is identified and $E_{\bftheta_{0}}\left[|\ln p(X;\bftheta)|\right]<\infty,\forall\bftheta\in\Theta$, then $\ell(\bftheta)$ is uniquely maximized at $\bftheta=\bftheta_{0}$.
\end{lemma}

\begin{proof}
	By the strict version of Jensen's inequality, with $\theta\neq\theta_{0}$,
	\begin{equation*}
		\begin{aligned}
			\ell\left(\theta_{0}\right)-\ell(\theta)= & \bbE_{\theta_{0}}\left\{-\ln\left[\frac{p(z\mid\theta)}{p(z\mid\theta_{0})}\right]\right\}>-\ln\bbE_{\theta_{0}}\left[\frac{p(z\mid\theta)}{p(z\mid\theta_{0})}\right] \\
			=                                         & -\ln \left[\int f(z \mid \theta) \dif z\right]=0
		\end{aligned}
	\end{equation*}
\end{proof}

\begin{theorem}[Consistency of MLE]
	Under the Assumption (1)-(4), we have
	\begin{equation}
		\hat{\theta}\stackrel{p}{\rightarrow}\theta_{0}
	\end{equation}
\end{theorem}

\begin{proof}
	Suppose
	\begin{equation*}
		\Theta(\epsilon)=\left\{\theta:\left\|\theta-\theta_{0}\right\|<\epsilon\right\},\quad\forall\varepsilon>0
	\end{equation*}

	Since $Q_{0}(\theta)$ is a continuous function, thus
	\begin{equation*}
		\theta^{*}:=\sup_{\theta\in\Theta\cap \Theta(\epsilon)^{C}}\left\{\ell(\theta)\right\}
	\end{equation*}
	is a achieved for a $\theta$ in the compact set $\theta\in\Theta\cap \Theta(\epsilon)^{C}$ (For open set $\Theta(\epsilon)$, $\Theta\cap\Theta(\epsilon)^{C}$ is a compact set). And $\theta_{0}$ is the unique maximized,
	\begin{equation*}
		\exists\delta>0,\quad\ell\left(\theta_{0}\right)-\ell\left(\theta^{*}\right)=\delta
	\end{equation*}

	\begin{enumerate}
		\item For $\theta\in\Theta\cap\Theta(\epsilon)^{C}$. suppose
		      \begin{equation*}
			      A_{n}=\left\{\sup_{\theta\in\Theta\cap\Theta(\epsilon)^{C}}\left|\hat{\ell}\left(\theta;\textbf{X}_{n}\right)-\ell(\theta)\right|<\frac{\delta}{2}\right\}
		      \end{equation*}
		      then,
		      \begin{equation*}
			      A_{n}\Longrightarrow\hat{\ell}\left(\theta;\textbf{X}_{n}\right)<\ell(\theta)+\frac{\delta}{2}\leq \ell\left(\theta^{*}\right)+\frac{\delta}{2}=\ell\left(\theta_{0}\right)-\frac{\delta}{2}
		      \end{equation*}
		\item For $\theta\in\Theta(\epsilon)$, suppose
		      \begin{equation*}
			      B_{n}=\left\{\sup_{\theta\in\Theta(\epsilon)}\left|\hat{\ell}\left(\theta\right)-\ell(\theta)\right|<\frac{\delta}{2}\right\}
		      \end{equation*}
		      then
		      \begin{equation*}
			      B_{n}\Longrightarrow\forall\theta\in\Theta(\epsilon),\,\hat{\ell}\left(\theta\right)>\ell(\theta)-\frac{\delta}{2}
		      \end{equation*}
	\end{enumerate}

	By the uniform law of large numbers, the dominance condition together with continuity establishes the uniform convergence in the probability of the log-likelihood:
	\begin{equation*}
		\sup_{\theta\in\Theta}|\hat{\ell}(\theta)-\ell(\theta)|\stackrel{p}{\rightarrow}0
	\end{equation*}
	Thus, we can conclude that
	\begin{equation*}
		P\left(A_{n}\cap B_{n}\right)\rightarrow 1
	\end{equation*}

	Within the definition
	\begin{equation*}
		\hat{\theta}=\max_{\theta\in\Theta}\hat{\ell}\left(\theta\right)
	\end{equation*}
	we have,
	\begin{equation*}
		A_{n}\cap B_{n}\Longrightarrow\hat{\theta}\in\Theta(\epsilon)
	\end{equation*}

	Hence,
	\begin{equation*}
		\forall\varepsilon>0,\,P\left[\hat{\theta}\in\Theta(\epsilon)\right]\rightarrow 1\Longrightarrow\hat{\theta}\stackrel{p}{\rightarrow}\theta_{0}
	\end{equation*}
\end{proof}

\subsection{Fisher Information}

\begin{definition}[Fisher Information]
	The Fisher information of a random variable $X$ with probability density function $p(X;\theta)$ is defined as
	\begin{equation}
		I(\theta)=\bbE\left[\left(\frac{\partial}{\partial\theta}\ln p(X;\theta)\right)^{2}\right].
	\end{equation}
\end{definition}

Alternatively, the Fisher information can be expressed as
\begin{equation}
	I(\theta)=-\bbE\left[\frac{\partial^{2}}{\partial\theta^{2}}\ln p(X;\theta)\right]
\end{equation}

\subsection{Asymptotic Normality}

\begin{enumerate}[label=(C\arabic*),resume]
	\item The information matrix $I(\bftheta)$ is positive definite.
	\item $\left\|\frac{\partial^{2}\log p(X;\bftheta)}{\partial\bftheta\partial\bftheta^{\top}}\right\|\leq M(x)$ for all $\bftheta\in\Theta$ and $\bbE_{\bftheta_{0}}M(x)<\infty$.
\end{enumerate}

\begin{proof}
	Since the MLE is the maximizer of the log-likelihood function, the score function evaluated at the MLE is zero, i.e., $\ell_{n}^{\prime}(\widehat{\bftheta})=\bfzero$. By the Taylor expansion of the score function around $\bftheta_{0}$, we have
	\begin{equation*}
		\bfzero=\ell_{n}^{\prime}(\widehat{\bftheta})=\ell_{n}^{\prime}\left(\bftheta_{0}\right)+\left(\widehat{\bftheta}-\bftheta_{0}\right)\ell_{n}^{\prime\prime}(\widetilde{\bftheta})
	\end{equation*}
	where $\tilde{\bftheta}$ lies between $\widehat{\bftheta}$ and $\bftheta_{0}$. Define $J_{n}(\bftheta)=-\frac{1}{n}\ell_{n}^{\prime\prime}(\bftheta)$. Then we have
	\begin{equation*}
		\sqrt{n}\left(\widehat{\bftheta}-\bftheta_{0}\right)=J_{n}^{-1}(\widetilde{\bftheta})n^{-1/2}\ell_{n}^{\prime}(\bftheta_{0}).
	\end{equation*}
	Then we need to show that: 1) $n^{-1/2}\ell_{n}^{\prime}(\bftheta_{0})\stackrel{d}{\rightarrow}\mcN\left(\bfzero, I(\bftheta_{0})\right)$; 2) $J_{n}(\widetilde{\bftheta})\stackrel{p}{\rightarrow}I(\bftheta_{0})$.

	For the first term, we have
	\begin{equation*}
		n^{-1/2}\ell_{n}^{\prime}(\bftheta_{0})=n^{-1/2}\sum_{i=1}^{n}\ell^{\prime}(X_{i};\bftheta_{0})=\sqrt{n}\frac{1}{n}\sum_{i=1}^{n}\frac{\partial\log p(X_{i};\bftheta_{0})}{\partial\bftheta},
	\end{equation*}
	thus, by the central limit theorem, we have
	\begin{equation*}
		\sqrt{n}\frac{1}{n}\sum_{i=1}^{n}\frac{\partial\log p(X_{i};\bftheta_{0})}{\partial\bftheta}\stackrel{d}{\rightarrow}\mcN\left(\bfzero,I(\bftheta_{0})\right).
	\end{equation*}

	For the second term, denote $I_{0}^{*}(\bftheta)=\bbE_{\bftheta_{0}}\left[-\frac{\partial^{2}\log p(X;\bftheta)}{\partial\bftheta\partial\bftheta^{\top}}\right]$, by the triangle inequality, we have
	\begin{equation*}
		\left\|J_{n}(\widetilde{\bftheta})-I(\bftheta_{0})\right\|\leq\left\|J_{n}(\widetilde{\bftheta})-I_{0}^{*}(\widetilde{\bftheta})\right\|+\left\|I_{0}^{*}(\widetilde{\bftheta})-I(\bftheta_{0})\right\|.
	\end{equation*}
	According to Lemma, we have
	\begin{equation*}
		\left\|J_{n}(\widetilde{\bftheta})-I_{0}^{*}(\widetilde{\bftheta})\right\|\leq\sup_{\bftheta\in\Theta}\left\|J_{n}(\bftheta)-I_{0}^{*}(\bftheta)\right\|\stackrel{p}{\rightarrow}0.
	\end{equation*}
	Since $I_{0}^{*}(\bftheta)$ is continuous in $\bftheta$, and $\widetilde{\bftheta}\stackrel{p}{\rightarrow}\bftheta_{0}$, by the continuous mapping theorem, we have
	\begin{equation*}
		I_{0}^{*}(\widetilde{\bftheta})\stackrel{p}{\rightarrow}I(\bftheta_{0}).
	\end{equation*}
	Combining the above results, we have
	\begin{equation*}
		J_{n}(\widetilde{\bftheta})\stackrel{p}{\rightarrow}I(\bftheta_{0}).
	\end{equation*}
	which completes the proof.
\end{proof}

\subsection{Efficiency}

\section{Modified Likelihood Estimator}

Seek a modified likelihood function that depends on as few of the nuisance parameters as possible while sacrificing as little information as possible.

\subsection{Marginal Likelihood}

\subsection{Conditional Likelihood}

Let $\boldsymbol{\theta}=(\boldsymbol{\varphi},\bfLambda)$, where $\boldsymbol{\varphi}$ is the parameter vector of interest and $\bfLambda$ is a vector of nuisance parameters. The conditional likelihood can be obtained as follows:
\begin{enumerate}
	\item Find the complete sufficient statistic $S_{\bfLambda}$, respectively for $\bfLambda$.
	\item  Construct the conditional log-likelihood
	      \begin{equation}
		      \ell_{c}=\ln\left(f_{Y\mid S_{\bfLambda}}\right)
	      \end{equation}
	      where $f_{Y\mid S_{\bfLambda}}$ is the conditional distribution of the response $Y$ given $S_{\bfLambda}$.
\end{enumerate}

\begin{remark}
	Two cases might occur, that, for fixed $\boldsymbol{\varphi}_{0}$, $S_{\bfLambda}\left(\boldsymbol{\varphi}_{0}\right)$ depends on $\boldsymbol{\varphi}_{0}$; or $S_{\bfLambda}\left(\boldsymbol{\varphi}_{0}\right)=S_{\bfLambda}$ is independent of $\boldsymbol{\varphi}_{0}$.
	\begin{enumerate}
		\item Independent:
		\item Dependent:
	\end{enumerate}
\end{remark}

Suppose that the log-likelihood for $\boldsymbol{\theta}=\left(\boldsymbol{\varphi},\bfLambda\right)$ can be written in the exponential family form
\begin{equation}
	\ell\left(\boldsymbol{\theta},\mathbf{y}\right)=\boldsymbol{\theta}^{\prime}\mathbf{s}-b\left(\boldsymbol{\theta}\right)
\end{equation}

Also, suppose $\ell\left(\boldsymbol{\theta},\mathbf{y}\right)$ has a decomposition of the form
\begin{equation}
	\ell\left(\boldsymbol{\theta},\mathbf{y}\right)=\boldsymbol{\varphi}^{\prime}\mathbf{s}_{1}+\bfLambda^{\prime}\mathbf{s}_{2}-b(\boldsymbol{\varphi},\bfLambda)
\end{equation}

\begin{remark}
	The above decomposition can be achieved only if $\boldsymbol{\varphi}$ is a linear function of $\theta$. The choice of nuisance parameter $\lambda$ is arbitrary and the inferences regarding $\boldsymbol{\varphi}$ should be unaffected by the parameterization chosen for $\lambda$.
\end{remark}

The conditional likelihood of the data $\mathbf{Y}$ given $\mathbf{s}_{2}$ is
\begin{equation}
	\ell\left(\boldsymbol{\varphi}\mid\mathbf{s}_{2}\right)=\boldsymbol{\varphi}^{\prime}\mathbf{s}_{1}-b^{*}\left(\boldsymbol{\varphi},\bfLambda\right)
\end{equation}
which is independent of the nuisance parameter and may be used for inferences regarding $\boldsymbol{\varphi}$.

\begin{example}
	$Y_{1}\sim P\left(\mu_{1}\right),Y_{2}\sim P\left(\mu_{2}\right)$ are independent. Suppose $\varphi=\ln\left(\frac{\mu_{2}}{\mu_{1}}\right)=\ln\left(\mu_{2}\right)-\ln\left(\mu_{1}\right)$ is the parameter of interest and the nuisance parameter is
	\begin{enumerate}
		\item $\lambda_{1}=\ln\left(\mu_{1}\right)$.
		\item
	\end{enumerate}
	Then, give the conditional log-likelihood for different nuisance parameters.
\end{example}

\begin{proof}
	\begin{enumerate}
		\item
		      The log-likelihood function in the form of $\left(\varphi,\lambda\right)$ is
		      \begin{equation*}
			      \begin{aligned}
				      \ell\left(\phi,\lambda_{1}\right)\propto & \ln\left[e^{-\left(\mu_{1}+\mu_{2}\right)}\mu_{1}^{y_{1}}\mu_{2}^{y_{2}}\right]                               \\
				      =                                        & -\left(\mu_{1}+\mu_{2}\right)+y_{1}\ln\left(\mu_{1}\right)+y_{2}\ln\left(\mu_{2}\right)                       \\
				      =                                        & -\mu_{1}\left(1+\frac{\mu_{2}}{\mu_{1}}\right)+y_{1}\ln\left(\mu_{1}\right)+y_{2}\ln\left(\mu_{1}\right)      \\
				                                               & -y_{2}\left[\ln\left(\mu_{1}\right)-\ln\left(\mu_{2}\right)\right]                                            \\
				      =                                        & -\mathrm{e}^{\lambda_{1}}\left(1+\mathrm{e}^{\varphi}\right)+\left(y_{1}+y_{2}\right)\lambda_{1}-y_{2}\varphi \\
				      =                                        & s_{1}\varphi+s_{2}\lambda_{1}-b\left(\varphi,\lambda_{1}\right)
			      \end{aligned}
		      \end{equation*}
		      where $s_{1}=-y_{2},s_{2}=y_{1}+y_{2},b\left(\varphi,\lambda_{1}\right)=e^{\lambda_{1}}\left(1+e^{\varphi}\right)$.

		      Then, the conditional distribution of $Y_{1},Y_{2}$ given $S_{2}=Y_{1}+Y_{2}$ is $b\left(S_{2},\frac{\mu_{1}}{\mu_{1}+\mu_{2}}\right)$, thus,
		      \begin{equation*}
			      \begin{aligned}
				      \ell\left(\varphi\mid S_{2}=s_{2}\right)\propto & y_{1}\ln\left(\frac{\mu_{1}}{\mu_{1}+\mu_{2}}\right)+y_{2}\ln\left(\frac{\mu_{2}}{\mu_{1}+\mu_{2}}\right)          \\
				      =                                               & y_{1}\ln\left(\frac{\mu_{1}}{\mu_{1}+\mu_{2}}\right)+y_{2}\ln\left(\frac{\mu_{1}}{\mu_{1}+\mu_{2}}\right)          \\
				                                                      & -y_{2}\left[\ln\left(\frac{\mu_{1}}{\mu_{1}+\mu_{2}}\right)-\ln\left(\frac{\mu_{2}}{\mu_{1}+\mu_{2}}\right)\right] \\
				      =                                               & \left(y_{1}+y_{2}\right)\ln\left(\frac{1}{1+e^{\varphi}}\right)-y_{2}\varphi                                       \\
				      =                                               & s_{1}\varphi-b^{*}\left(\varphi,s_{2}\right)
			      \end{aligned}
		      \end{equation*}
		      where $b^{*}\left(\varphi,s_{2}\right)=-s_{2}\ln\left(\frac{1}{1+\varphi^{-1}}\right)$.
	\end{enumerate}
\end{proof}

\subsection{Profile Likelihood}

\subsection{Quasi Likelihood}

\section{Minimum-Variance Unbiased Estimator}

\begin{definition}[UMVU Estimators]
	An unbiased estimator $\delta(\textbf{X})$ of $g(\theta)$ is the uniform minimum variance unbiased (UMVU) estimator of $g(\theta)$ if
	\begin{equation}
		\Var_{\theta}\delta(\textbf{X})\leq\Var_{\theta}\delta'(\textbf{X}),\quad\forall\theta\in\Theta,
	\end{equation}
	where $\delta'(\textbf{X})$ is any other unbiased estimator of $g(\theta)$.
\end{definition}

\begin{remark}
	If there exists an unbiased estimator of $g$, the estimand $g$ will be called $U$-estimable.
\end{remark}

\begin{enumerate}
	\item If $T(\textbf{X})$ is a complete sufficient statistic, estimator $\delta(\textbf{X})$ that only depends on $T(\textbf{X})$, then for any $U$-estimable function $g(\theta)$ with
	      \begin{equation}
		      E_{\theta}\delta(T(\textbf{X}))=g(\theta),\quad\forall\theta\in\Theta,
	      \end{equation}
	      hence, $\delta(T(\textbf{X}))$ is the unique UMVU estimator of $g(\theta)$.
	\item If $T(\textbf{X})$ is a complete sufficient statistic and $\delta({\textbf{X}})$ is any unbiased estimator of $g(\theta)$, then the UMVU estimator of $g(\theta)$ can be obtained by
	      \begin{equation}
		      E\left[\delta(\textbf{X})\mid T(\textbf{X})\right].
	      \end{equation}
\end{enumerate}

\begin{example}[Estimating Polynomials of a Normal Variance]
	Let $X_{1},\ldots,X_{n}$ be distributed with joint density
	\begin{equation}
		\frac{1}{(\sqrt{2\pi}\sigma)^{n}}\exp\left[-\frac{1}{2\sigma^{2}}\sum\left(x_{i}-\xi\right)^{2}\right].
	\end{equation}
	Discussing the UMVU estimators of $\xi^r$, $\sigma^r$, $\xi/\sigma$.
\end{example}

\begin{proof}
	\begin{enumerate}
		\item \textbf{$\sigma$ is known}:

		      Since $\bar{X}=\frac{1}{n}\sum_{i=1}^{n}X_i$ is the complete sufficient statistic of $X_i$, and
		      \begin{equation*}
			      E(\bar{X})=\xi,
		      \end{equation*}
		      then the UMVU estimator of $\xi$ is $\bar{X}$.

		      Therefore, the UMVU estimator of $\xi^r$ is $\bar{X}^r$ and the UMVU estimator of $\xi/\sigma$ is $\bar{X}/\sigma$.

		\item \textbf{$\xi$ is known}:

		      Since $s^r=\sum\left(x_{i}-\xi\right)^r$ is the complete sufficient statistic of $X_i$.

		      Assume
		      \begin{equation*}
			      E\left[\frac{s^r}{\sigma^r}\right]=\frac{1}{K_{n,r}},
		      \end{equation*}
		      where $K_{n,r}$ is a constant depends on $n,r$.

		      Since $s^2/\sigma^2\sim\text{Ga}(n/2,1/2)=\chi^2(n)$, then
		      \begin{equation*}
			      E\left[\frac{s^r}{\sigma^r}\right]=E\left[\left(\frac{s^2}{\sigma^2}\right)^{\frac{r}{2}}\right]=\int_{0}^{\infty}x^{\frac{r}{2}}\frac{1}{2^{\frac{n}{2}}\Gamma(\frac{n}{2})}x^{\frac{n}{2}-1}e^{-\frac{x}{2}}\dif x=\frac{\Gamma\left(\frac{n+r}{2}\right)}{\Gamma(\frac{n}{2})}\cdot 2^{\frac{r}{2}}.
		      \end{equation*}
		      therefore,
		      \begin{equation*}
			      K_{n,r}=\frac{\Gamma(\frac{n}{2})}{2^{\frac{r}{2}}\cdot\Gamma\left(\frac{n+r}{2}\right)}.
		      \end{equation*}

		      Hence,
		      \begin{equation*}
			      E\left[s^rK_{n,r}\right]=\sigma^r \text{ and } E[\xi s^{-1}K_{n,-1}]=\xi/\sigma,
		      \end{equation*}
		      which means the UMVU estimator of $\sigma^r$ is $s^rK_{n,r}$ and the UMVU estimator of $\xi/\sigma$ is $\xi s^{-1}K_{n,-1}$.

		\item \textbf{Both $\xi$ and $\sigma$ is unknown}:

		      Since $(\bar{X},s_x^r)$ are the complete sufficient statistic of $X_i$, where $s_x^2=\sum\left(x_{i}-\bar{X}\right)^r$.

		      Since $s_x^2/\sigma^2\sim\chi^2(n-1)$, then
		      \begin{equation*}
			      E\left[\frac{s_x^r}{\sigma^r}\right]=\frac{1}{K_{n-1,r}}.
		      \end{equation*}

		      Hence,
		      \begin{equation*}
			      E\left[s_x^rK_{n-1,r}\right]=\sigma^r,
		      \end{equation*}
		      which means the UMVU estimator of $\sigma^r$ is $s_x^rK_{n-1,r}$,
		      and
		      \begin{equation*}
			      E(\bar{X}^r)=\xi^r,
		      \end{equation*}
		      which means the UMVU estimator of $\xi^r$ is $\bar{X}^r$.

		      Since $\bar{X}$ and $s_x^r$ are independent, then
		      \begin{equation*}
			      E[\bar{X}s_x^{-1}K_{n-1,-1}]=\xi/\sigma
		      \end{equation*}
		      which means the UMVU estimator of $\xi/\sigma$ is $\bar{X}s_x^{-1}K_{n-1,-1}$.
	\end{enumerate}
\end{proof}

\begin{example}[]
	Let $X_{1},\ldots,X_{n}$ be i.i.d sample from $U\left(\theta_1-\theta_2,\theta_1+\theta_2\right)$, where $\theta_1\in\bbR,\theta_2\in\bbR^+$. Discussing the UMVU estimators of $\theta_1,\theta_2$.
\end{example}

\begin{proof}
	Let $X_{(i)}$ be the $i$-th order statistic of $X_i$, then $\left(X_{(1)},X_{(n)}\right)$ is the complete and sufficient statistic for $(\theta_1,\theta_2)$. Thus it suffices to find a function $\left(X_{(1)},X_{(n)}\right)$, which is unbiased of $(\theta_1,\theta_2)$.

	Let
	\begin{equation*}
		Y_i=\frac{X_i-(\theta_1-\theta_2)}{2\theta_2}\sim U(0,1),
	\end{equation*}
	and
	\begin{equation*}
		Y_{(i)}=\frac{X_{(i)}-(\theta_1-\theta_2)}{2\theta_2},
	\end{equation*}
	be the $i$-th order statistic of $Y_i$, then we got
	\begin{equation*}
		\begin{aligned}
			E[X_{(1)}] & = 2\theta_2E[Y_{(1)}]+(\theta_1-\theta_2)                      \\
			           & = 2\theta_2\int_{0}^{1}ny(1-y)^{n-1}\dif y+(\theta_1-\theta_2) \\
			           & = \theta_1-\frac{3n+1}{n+1}\theta_2                            \\
			E[X_{(n)}] & = 2\theta_2E[Y_{(n)}]+(\theta_1-\theta_2)                      \\
			           & = 2\theta_2\int_{0}^{1}ny^{n}\dif y+(\theta_1-\theta_2)        \\
			           & = \theta_1+\frac{n-1}{n+1}\theta_2                             \\
		\end{aligned}.
	\end{equation*}

	Thus,
	\begin{equation*}
		\begin{aligned}
			\theta_1 & = E\left[\frac{n-1}{4n}X_{(1)}+\frac{3n+1}{4n}X_{(n)}\right], \\
			\theta_2 & = E\left[-\frac{n+1}{4n}X_{(1)}+\frac{n+1}{4n}X_{(n)}\right], \\
		\end{aligned}
	\end{equation*}
	which means the UMVU estimator is
	\begin{equation*}
		\hat{\theta_1}=\frac{n-1}{4n}X_{(1)}+\frac{3n+1}{4n}X_{(n)},\quad\hat{\theta_2}=-\frac{n+1}{4n}X_{(1)}+\frac{n+1}{4n}X_{(n)}.
	\end{equation*}
\end{proof}

\section{Accuracy of Estimators}

\begin{example}[Normal Probability]
	Let $X_{1},\ldots,X_{n}$ be i.i.d. as $\mcN(\theta,1)$ and consider the estimation of $p=P\left(X_{i}\leq u\right)=\Phi(u-\theta)$. The maximum likelihood estimator of $p$ is $\hat{p}=\Phi(u-\bar{X})$, and we shall attempt to obtain large-sample approximations for the bias and variance of this estimator.
\end{example}

\begin{proof}
	Since $\bar{X}-\theta$ is likely to be small, it is natural to write
	\begin{equation*}
		\Phi(u-\bar{X})=\Phi[(u-\theta)-(\bar{X}-\theta)]
	\end{equation*}
	and to expand the right side about $u-\theta$ by Taylor's theorem as
	\begin{equation}
		\label{eq:taylor-expansion-normal-probability}
		\begin{aligned}
			\Phi(u-\bar{X})= & \Phi(u-\theta)-(\bar{X}-\theta)\phi(u-\theta)+\frac{1}{2}(\bar{X}-\theta)^{2}\phi^{\prime}(u-\theta)                                  \\
			                 & \qquad -\frac{1}{6}(\bar{X}-\theta)^{3}\phi^{\prime\prime}(u-\theta)+\frac{1}{24}(\bar{X}-\theta)^{4}\phi^{\prime\prime \prime}(\xi),
		\end{aligned}
	\end{equation}
	where $\xi$ is a random quantity that lies between $u-\theta$ and $u-\bar{X}$.

	To calculate the bias
	\begin{equation*}
		\bbE[\Phi(u-\bar{X})]-\Phi(u-\theta),
	\end{equation*}
	we take the expectation of \eqref{eq:taylor-expansion-normal-probability}, which yields
	\begin{equation*}
		\bbE[\Phi(u-\bar{X})]=p+\frac{1}{2n}\phi^{\prime}(u-\theta)+\frac{1}{24}\bbE\left[(\bar{X}-\theta)^4\phi^{\prime\prime\prime}(\xi)\right].
	\end{equation*}
	Since the derivatives of $\phi(x)$ all are of the form $P(x)\phi(x)$, where $P(x)$ is a polynomial in $x$ and are therefore all bounded. It follows that
	\begin{equation*}
		\left|\bbE(\bar{X}-\theta)^{4}\phi^{\prime\prime\prime}(\xi)\right|<M\bbE(\bar{X}-\theta)^{4}=3M/n^2,
	\end{equation*}
	for some finite $M$. Using the fact that $\phi^{\prime}(x)=-x \phi(x)$, we therefore find that
	\begin{equation*}
		\bbE(\hat{p})=p-\frac{1}{2n}(u-\theta)\phi(u-\theta)+O\left(1/n^{2}\right),
	\end{equation*}
	where the error term is uniformly $O\left(1 / n^2\right)$.
	The estimator $\delta$ therefore has a bias of order $1/n$ which tends to zero as $\theta \rightarrow \pm \infty$.

	In the same way, one can show that (Problem (2.2(i))
	\begin{equation*}
		\Var(\delta)=\frac{1}{n} \phi^2(u-\theta)+O\left(\frac{1}{n^2}\right)
	\end{equation*}
	and hence that
	\begin{equation*}
		\Var(\sqrt{n} \delta) \rightarrow \phi^2(u-\theta)
	\end{equation*}

	Since (Problem 2.2(ii))
	\begin{equation*}
		\sqrt{n}[\delta-\Phi(u-\theta)] \xrightarrow{L} N\left(0, \phi^2(u-\theta)\right),
	\end{equation*}
	the limit of the variance in this case is equal to the asymptotic variance.

	It is interesting to see what happens if the expansion (4.2.13) is carried one step less far. Then
	\begin{equation*}
		E[\Phi(u-\bar{X})]=p+\frac{1}{2 n} \phi^{\prime}(u-\theta)+\frac{1}{6} E\left[(\bar{X}-\theta)^3 \phi^{\prime \prime}(\xi)\right] .
	\end{equation*}

	Since the third derivative of $\phi$ is bounded, the remainder now satisfies
	\begin{equation*}
		\frac{1}{6} E\left|(\bar{X}-\theta)^3 \phi^{\prime \prime \prime}(\xi)\right|<M^{\prime} E|\bar{X}-\theta|^3=O\left(\frac{1}{n^{3 / 2}}\right) .
	\end{equation*}

	The conclusion is therefore weaker than before.
\end{proof}
