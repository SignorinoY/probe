\chapter{Introduction}

\section{Populations and Samples}

\section{Statistics}

\subsection{Sufficient Statistics}

\begin{definition}{Sufficient Statistics}{}
    A statistic $T$ is said to be sufficient for $X$, or for the family $\mathcal{P}=\left\{P_{\theta}, \theta \in \Omega\right\}$ of possible distributions of $X$, or for $\theta$, if the conditional distribution of $X$ given $T=t$ is independent of $\theta$ for all $t$.
\end{definition}

\begin{theorem}{Fisher–Neyman Factorization Theorem}{}
    If the probability density function is $p_{\theta}(x)$, then $T$ is sufficient for $\theta$ if and only if nonnegative functions $g$ and $h$ can be found such that
    \begin{equation*}
        p_{\theta}(x)=h(x)g_{\theta}[T(x)].
    \end{equation*}
\end{theorem}

\begin{proof}

\end{proof}

\subsection{Complete Statistics}

\begin{definition}{Complete Statistics}{}
    A statistic $T$ is said to be complete, if $Eg(T)=0$ for all $\theta$ and some function $g$ implies that $P(g(T)=0\mid\theta)=1$ for all $\theta$.
\end{definition}

\section{Estimators}

\subsection{Definition of Estimators}

\begin{definition}{Estimator}{estimator}
    An estimator is a real-valued function defined over the sample space, that is
    \begin{equation}
        \delta:\textbf{X}\rightarrow\mathbb{R}.
    \end{equation}
    It is used to estimate an estimand, $\theta$, a real-valued function of the parameter.
\end{definition}

\subsection{Properties of Estimators}

\subsubsection*{Unbiasedness}

\begin{definition}{Unbiasedness}{}
    An estimator $\hat{\theta}$ of $\theta$ is unbiased if
    \begin{equation}
        E\hat{\theta}=\theta,\quad\forall\theta\in\Theta.
    \end{equation}
\end{definition}

\begin{note}
    \begin{itemize}
        \item Unbiased estimators of $\theta$ may not exist.
        \item
    \end{itemize}
\end{note}

\begin{example}[Nonexistence of Unbiased Estimator]

\end{example}

\subsubsection*{Consistency}

\begin{definition}{Consistency}{}
    An estimator $\hat{\theta}_n$ of $\theta$ is consistent if
    \begin{equation}
        \lim_{n\rightarrow\infty}P\left(\left|\hat{\theta}_n-\theta\right|>\varepsilon\right)=0,\quad\forall\varepsilon>0.
    \end{equation}
\end{definition}

\begin{example}[Unbiased But Consistent]

\end{example}

\begin{example}[Biased But Not Consistent]

\end{example}

\subsubsection*{Asymptotic Normality}

\begin{definition}{Asymptotic Normality}{}
    An estimator $\hat{\theta}_n$ of $\theta$ is asymptotic normality if
    \begin{equation}
        \sqrt{n}\left(\hat{\theta}-\theta\right)\stackrel{d}{\rightarrow}N\left(0,\sigma_{\theta}^{2}\right).
    \end{equation}
\end{definition}

\subsubsection*{Efficiency}

\begin{definition}{Efficiency}{}

\end{definition}

\subsubsection*{Robustness}

\begin{definition}{Robustness}{}

\end{definition}
