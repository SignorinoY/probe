\chapter{Multivariate Extensions}

\section{Applications}

\subsection{Mean vector}

Let $\bfX_{i}$, $i=1,\ldots,n$ be drawn from a $p$-dimensional distribution with mean vector $\bfmu$ and covariance matrix $\bfSigma$. The sample mean vector is given by
\begin{equation*}
	\bar{\bfX}=\frac{1}{n}\sum_{i=1}^{n}\bfX_{i}.
\end{equation*}

By the multivariate central limit theorem, we have
\begin{equation*}
	\sqrt{n}(\bar{\bfX}-\bfmu)\xrightarrow{d}\mcN(\bfzero,\bfSigma).
\end{equation*}
Applying the continuous mapping theorem and Theorem \ref{thm:mvn-properties}, we obtain
\begin{equation*}
	n(\bar{\bfX}-\bfmu)^{\top}\bfSigma^{-1}(\bar{\bfX}-\bfmu)\sim\chi^{2}_{p}.
\end{equation*}

If $\bfSigma$ is known, then the confidence region for $\bfmu$ with confidence level $1-\alpha$ is given by
\begin{equation*}
	n(\bar{\bfX}-\bfmu)^{\top}\bfSigma^{-1}(\bar{\bfX}-\bfmu)\leq C_{p},
\end{equation*}
where
\begin{equation*}
	\int_{0}^{C_{p}}\chi^{2}_{p}(t)\dif t=1-\alpha.
\end{equation*}

In applications where $\bfSigma$ is unknown, we can use the sample covariance matrix $\widehat{\bfSigma}$ to replace $\bfSigma$, and $\widehat{\bfSigma}^{-1}$ is a consistent estimator of $\bfSigma^{-1}$. This leads to the following confidence region for $\bfmu$:
\begin{equation*}
	n(\bar{\bfX}-\bfmu)^{\top}\widehat{\bfSigma}^{-1}(\bar{\bfX}-\bfmu)\leq C_{p},
\end{equation*}
which provides asymptotically valid confidence regions for $\bfmu$ at level $1-\alpha$ for any non-singular $\bfSigma$ and fixed shape of distribution $F$.

\begin{proof}
	To show that
	\begin{equation*}
		\Pr\left(n(\bar{\bfX}-\bfmu)^{\top}\bfSigma^{-1}(\bar{\bfX}-\bfmu)\leq C_{p}\right)\rightarrow1-\alpha,\quad\text{ as }n\rightarrow\infty.
	\end{equation*}
\end{proof}

\begin{remark}
	If $\bfX_{i}$, $i=1,\ldots,n$ are drawn from a $p$-dimensional multivariate normal distribution, the exact distribution of $n(\bar{\bfX}-\bfmu)^{\top}\bfSigma^{-1}(\bar{\bfX}-\bfmu)$ is a Hotelling's $T^{2}$ distribution with parameters $p$ and $n-1$.
\end{remark}

We next consider the power of

\subsection{Difference of two mean vectors}

\begin{equation*}
	T_{mn}=\left[(\bfeta-\bfxi)-(\bar{\bfY}-\bar{\bfX})\right]^{\top}\left(\frac{1}{m}\hat{\bfSigma}+\frac{1}{n}\hat{\bfGamma}\right)^{-1}[(\bfeta-\bfxi)-(\bar{\bfY}-\bar{\bfX})]\leq C_{p},
\end{equation*}

Univariate case:
\begin{equation*}
	T_{mn}=\frac{(\bar{Y}-\bar{X})/\sqrt{\frac{1}{m}+\frac{1}{n}}}{\sqrt{\left[\sum_{i}\left(X_i-\bar{X}\right)^2+\sum_{j}\left(Y_j-\bar{Y}\right)^2\right]/(m+n-2)}}
\end{equation*}

Since
\begin{equation*}
	\frac{(\bar{Y}-\bar{X})-(\eta-\xi)}{\sqrt{\frac{\sigma^{2}}{m}+\frac{\gamma^{2}}{n}}}\sim\mathcal{N}(0,1)
\end{equation*}
and
\begin{equation*}
	\frac{\sum_{i}\left(X_i-\bar{X}\right)^2}{\sigma^{2}}+\frac{\sum_{j}\left(Y_j-\bar{Y}\right)^2}{\gamma^{2}}\sim\chi^{2}_{m+n-2},
\end{equation*}
then, we have
\begin{equation*}
	T_{mn}^{\prime}=\frac{\frac{(\bar{Y}-\bar{X})-(\eta-\xi)}{\sqrt{\frac{\sigma^{2}}{m}+\frac{\gamma^{2}}{n}}}}{\sqrt{\left(\frac{\sum_{i}\left(X_i-\bar{X}\right)^2}{\sigma^{2}}+\frac{\sum_{j}\left(Y_j-\bar{Y}\right)^2}{\gamma^{2}}\right)/(m+n-2)}}\sim t_{m+n-2}.
\end{equation*}

Under the null hypothesis, we have
\begin{equation*}
	\begin{aligned}
		T_{mn}^{\prime}= & T_{mn}\cdot\sqrt{\frac{\sum_{i}\left(X_i-\bar{X}\right)^2+\sum_{j}\left(Y_j-\bar{Y}\right)^2}{\gamma^{2}\sum_{i}\left(X_i-\bar{X}\right)^2+\sigma^{2}\sum_{j}\left(Y_j-\bar{Y}\right)^2}}\cdot\sqrt{\frac{(m+n)\sigma^{2}\gamma^{2}}{n\sigma^{2}+m\gamma^{2}}}.
	\end{aligned}
\end{equation*}
It is easy to see that
\begin{gather*}
	\sum_{i}\left(X_i-\bar{X}\right)^2+\sum_{j}\left(Y_j-\bar{Y}\right)^2\rightarrow_{p} m\sigma^{2}+n\gamma^{2},\\
	\gamma^{2}\sum_{i}\left(X_i-\bar{X}\right)^2+\sigma^{2}\sum_{j}\left(Y_j-\bar{Y}\right)^2\rightarrow_{p} (m+n)\sigma^{2}\gamma^{2}>0,
\end{gather*}
and by the continuous mapping theorem with the fact that $m/(m+n)\rightarrow\rho$, we have
\begin{equation*}
	\sqrt{\frac{\sum_{i}\left(X_i-\bar{X}\right)^2+\sum_{j}\left(Y_j-\bar{Y}\right)^2}{\gamma^{2}\sum_{i}\left(X_i-\bar{X}\right)^2+\sigma^{2}\sum_{j}\left(Y_j-\bar{Y}\right)^2}}\cdot\sqrt{\frac{(m+n)\sigma^{2}\gamma^{2}}{n\sigma^{2}+m\gamma^{2}}}\rightarrow_{p}\sqrt{\frac{\rho\sigma^{2}+(1-\rho)\gamma^{2}}{(1-\rho)\sigma^{2}+\rho\gamma^{2}}}.
\end{equation*}

\begin{proof}
	\begin{equation*}
		\Pr\left(T_{mn}\leq C_{k}\right)\rightarrow\gamma,\quad\text{ as }m\text{ and }n\rightarrow\infty
	\end{equation*}

	For convenience, suppose $m<n$, then we define
	\begin{equation*}
		\bfZ_{i}=\bfX_{i}-\sqrt{\frac{m}{n}}Y_{i}+\frac{m}{\sqrt{mn}}
	\end{equation*}
\end{proof}

\subsection{Simple Linear Regression}

Suppose
\begin{equation*}
	\bfX_{i}=\bfalpha+\bfv_{i}\bfbeta+\bfvareps_{i},\quad i=1,\ldots,n
\end{equation*}
where $\bfv_{i}$ is a $p$-dimensional vector of known constants, and $\bfvareps_{i}$ is a $p$-dimensional random vector with mean $\bfzero$ and covariance matrix $\bfSigma$. The least squares estimator of $\bfalpha$ and $\bfbeta$ are given by
\begin{equation*}
	\hat{\beta}_{j}=\frac{\sum_{i}(X_{i,j}-\bar{X}_{j})(v_{i,j}-\bar{v}_{j})}{\sum_{i}(v_{i,j}-\bar{v}_{j})^{2}},\quad\hat{\alpha}_{j}=\bar{X}_{j}-\hat{\beta}_{j}\bar{v}_{j}.
\end{equation*}
where $\bar{\bfX}=\frac{1}{n}\sum_{i}\bfX_{i}$ and $\bar{\bfv}=\frac{1}{n}\sum_{i}\bfv_{i}$.

Denote
\begin{equation*}
	d_{n,j}^{(i)}=\frac{v_{i,j}-\bar{v}_{j}}{\sqrt{\sum_{i}(v_{i,j}-\bar{v}_{j})^{2}}}.
\end{equation*}
We can rewrite the least squares estimator as
\begin{equation*}
	\widehat{\beta}_{j}-\beta_{j}=\frac{\sum_{i}d_{n,j}^{(i)}\left[(X_{i,j}-\bar{X}_{j})-\bbE(X_{i,j}-\bar{X}_{j})\right]}{\sqrt{\sum_{i}(v_{i,j}-\bar{v}_{j})^{2}}},
\end{equation*}

By Theorem, we have
\begin{equation*}
	\left(\sqrt{\sum_{i}(v_{i,1}-\bar{v}_{1})^{2}}(\hat{\beta}_{1}-\beta_{1}),\ldots,\sqrt{\sum_{i}(v_{i,p}-\bar{v}_{p})^{2}}(\hat{\beta}_{p}-\beta_{p})\right)^{\top}\xrightarrow{d}\mcN(\bfzero,\bfSigma).
\end{equation*}

\subsection{Multinomial One-Sample Test}

Consider a sequence of $n$ independent multinomial trials with $k+1$ categories, where the probability of the $i$-th category is $p_{i}$, $i=1,\ldots,k+1$.

Let us now consider testing the hypothesis
\begin{equation*}
	H_{0}:p_i=p_i^{(0)},\quad i=1,\ldots,k+1
\end{equation*}
against the alternatives that $p_{i}\neq p_{i}^{(0)}$ for at least some $i$. The standard test for this problem is Pearson's $\chi^2$-test, which rejects $H_{0}$ when
\begin{equation*}
	Q=n\sum_{i=1}^{k+1}\left(\frac{Y_i}{n}-p_i^{(0)}\right)^2/p_i^{(0)}\geq C_k,
\end{equation*}

The asymptotic distribution of $Q$ is $\chi^{2}_{k}$ under $H_{0}$.

\begin{proof}
	It follows from (5.4.14) and Theorem \ref{thm:mvn-properties} that
	\begin{equation}
		\label{eq:multinomial-test-chi2}
		n\sum_{i=1}^{k}\sum_{j=1}^{k}a_{ij}\left(\frac{Y_i}{n}-p_i^{(0)}\right)\left(\frac{Y_j}{n}-p_j^{(0)}\right)\xrightarrow{L}\chi_k^2,
	\end{equation}
	where
	\begin{equation*}
		a_{ij}=\begin{cases}
			\frac{1}{p_{i}^{(0)}}+\frac{1}{p_{k+1}^{(0)}} & \text{if }i=j,     \\
			\frac{1}{p_{k+1}^{(0)}}                       & \text{if }i\neq j.
		\end{cases}
	\end{equation*}

	The left side of \eqref{eq:multinomial-test-chi2} can be written as
	\begin{equation*}
		n\sum_{i=1}^{k}\frac{1}{p_i^{(0)}}\left(\frac{Y_i}{n}-p_i^{(0)}\right)^2+\frac{n}{p_{k+1}^{(0)}}\sum_{i=1}^{k}\sum_{j=1}^k\left(\frac{Y_i}{n}-p_i^{(0)}\right)\left(\frac{Y_j}{n}-p_j^{(0)}\right),
	\end{equation*}
	The last term is equal to
	\begin{equation*}
		n\left[\sum_{i=1}^k\left(\frac{Y_i}{n}-p_i^{(0)}\right)\right]^2/p_{k+1}^{(0)}=n\left(\frac{Y_{k+1}}{n}-p_{k+1}^{(0)}\right)^2/p_{k+1}^{(0)},
	\end{equation*}
	and completes the proof.
\end{proof}

\subsection{Contingency Table}
