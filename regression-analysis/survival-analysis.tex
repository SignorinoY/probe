\chapter{Survival Analysis}

\section{General Formulation}

\begin{definition}[Survival Function]
    The survival function\footnote{The survival function is the probability that the time of death is later than some specified time $t$.} is defined to be
    \begin{equation}
        S(t)=P(T>t)=\int_{t}^{\infty}f(u)\,\mathrm{d}u=1-F(t) .
    \end{equation}
    where $t$ is some specified time, $T$ is a random variable denoting the time of death.
\end{definition}

\begin{definition}[Lifetime Distribution Function]
    The lifetime distribution function is defined to be
    \begin{equation}
        F(t)=P(T\leq t)
    \end{equation}
    If $F$ is differentiable then the derivative, which is the density function of the lifetime distribution\footnote{The function $f$ is sometimes called the event density; it is the rate of death or failure events per unit time.}, is defined to be
    \begin{equation}
        f(t)=F^{\prime}(t)=\frac{\mathrm{d}}{\mathrm{d}t}F(t)
    \end{equation}
\end{definition}

\begin{definition}[Hazard Function]
    The Hazard function\footnote{The Hazard function is the event rate at time $t$ conditional on survival until time $t$ or later (that is, $T\geq t$).} is defined to be
    \begin{equation}
        \lambda(t)=\lim_{\varepsilon\rightarrow 0^{+}}\left[\frac{P(t\leq T<t+\varepsilon\mid T\geq t)}{\varepsilon}\right]=\frac{f(t)}{S(t)}
    \end{equation}
\end{definition}

\begin{property}
    The relationship among $\lambda(t),f(t),S(t)$,
    \begin{enumerate}
        \item
              \begin{equation}
                  \lambda(t)=-\frac{\mathrm{d}\log [S(t)]}{\mathrm{d}t}
              \end{equation}
        \item
              \begin{equation}
                  S(t)=\exp\left[-\int_{0}^{t}\lambda(x)\,\mathrm{d}x\right]
              \end{equation}
        \item
              \begin{equation}
                  f(t)=\lambda(t)\exp\left[-\int_{0}^{t}\lambda(x)\,\mathrm{d}x\right]
              \end{equation}
    \end{enumerate}
\end{property}

\begin{proof}

\end{proof}

\begin{example}[Constant Hazards]
    Suppose
    \begin{equation}
        \lambda(t)=\lambda
    \end{equation}
    then
    \begin{gather*}
        S(t)=\exp\left[-\int_{0}^{t}\lambda(x)\,\mathrm{d}x\right]=\exp\left[-\int_{0}^{t}\lambda\,\mathrm{d}x\right]=\exp(-\lambda t) \\
        f(t)=\lambda(t)\exp\left[-\int_{0}^{t}\lambda(x)\mathrm{d}x\right]=\lambda\exp\left[-\int_{0}^{t}\lambda\mathrm{d}x\right]=\lambda\exp(-\lambda t)
    \end{gather*}
    which is the exponential distribution.
\end{example}

\begin{example}[Bathtub Hazards]
    \begin{equation}
        \lambda(t)=\alpha t+\frac{\beta}{1+\gamma t}
    \end{equation}
\end{example}

\section{Estimation of Survival Function}

\paragraph*{Parametric Approach}

Suppose $t_{1},t_{2},\ldots,t_{n}$ are failure times corresponding to censor indicators $\delta_{1},\delta_{2},\ldots,\delta_{n}$. The likelihood function is
\begin{equation}
    \begin{aligned}
        f\left(\boldsymbol{\theta}\mid\mathbf{t},\boldsymbol{\delta}\right)= & \prod_{i=1}^{n}\left[f\left(t_{i}\right)\right]^{\delta_{i}}\left[S\left(t_{i}\right)\right]^{1-\delta_{i}} \\
        =                                                                    & \prod_{i=1}^{n}\left(\frac{f\left(t_{i}\right)}{S\left(t_{i}\right)}\right)^{\delta_{i}}S\left(t_{i}\right) \\
        =                                                                    & \prod_{i=1}^{n}\left[\lambda\left(t_{i}\right)\right]^{\delta_{i}}S\left(t_{i}\right)
    \end{aligned}
\end{equation}
where $\lambda(t),S(t)$ depends on some parameter $\theta$.

\begin{example}
    Suppose $\boldsymbol{T}$ have exponential density, that,
    \begin{equation*}
        f(t)=\lambda \mathrm{e}^{-\lambda t},\quad S(t)=\mathrm{e}^{-\lambda t}
    \end{equation*}
    Thus,
    \begin{equation*}
        \begin{aligned}
            \ell(\lambda)= & \log[\ell(\theta)]=\sum_{i=1}^{n}\left[\delta_{i}\log(\lambda)-\lambda t_{i}\right]        \\
            =              & \left(\sum_{i=1}^{n}\delta_{i}\right)\log(\lambda)-\lambda\left(\sum_{i=1}^{n}t_{i}\right)
        \end{aligned}
    \end{equation*}
    Hence,
    \begin{equation*}
        \frac{\partial\ell(\lambda)}{\partial\lambda}=\frac{\sum_{i=1}^{n}\delta_{i}}{\lambda}-\sum_{i=1}^{n}t_{i}=0\Rightarrow\hat{\lambda}=\frac{\sum_{i=1}^{n}\delta_{i}}{\sum_{i=1}^{n}t_{i}}
    \end{equation*}
\end{example}

\paragraph*{Nonparametric Approach}

Then, for $t_{(k)}\leq t<t_{(k+1)}$,
\begin{equation}
    \begin{aligned}
        \hat{S}(t)= & \prod_{j=1}^{k}\left(\frac{n_{j}-d_{j}}{n_{j}}\right)                                                                                             \\
        =           & \left(1-\frac{d_{1}}{n_{1}}\right)\left(1-\frac{d_{2}}{n_{2}}\right) \cdots\left(1-\frac{d_{k}}{n_{k}}\right)                                     \\
        \approx     & \left[1-\hat{\lambda}\left(t_{1}\right)\right]\left[1-\hat{\lambda}\left(t_{2}\right)\right] \ldots\left[1-\hat{\lambda}\left(t_{k}\right)\right]
    \end{aligned}
\end{equation}
where $\hat{S}(t)$ is referred to as Kaplan-Meier estimate.

\section{Proportional Hazards Model}

Let $t_{1},t_{2},\ldots,t_{n}$ be the failure times associated with censor indicator $\delta_{1},\delta_{2},\ldots,\delta_{n}$ and the covariate vectors $\mathbf{x}_{i}$.

Further, let $t_{(1)}\leq t_{(2)}\leq\ldots\leq t_{(m)}$ be the ordered uncensored failure times corresponding to $\delta_{(j)}=1,j=1,2,\ldots,m$, and $x_{(1)},x_{(2)},\ldots,x_{(m)}$ are the associated covariate vectors. Note $(j)$ represents the label for the individual who dies at $t_{(j)}$.

The proportional hazards model specifying the hazard at time $t$ for an individual whose covariate vector is $\mathbf{x}$ is given by
\begin{equation}
    \lambda(t)=\lambda_{0}(t)\mathrm{e}^{\mathbf{x}^{\prime}\boldsymbol{\beta}}
\end{equation}
where $\lambda_{0}(t)$ is referred to as the bașeline hazard function.

The exact likelihood function is
\begin{equation}
    \ell\left[\beta,\lambda_{0}(t)\right]=\prod_{i=1}^{n}\left[\lambda_{i}\left(t_{i}\right)\right]^{\delta_{i}}S\left(t_{i}\right)
\end{equation}
depends on both the nonparametric function $\lambda_{0}(t)$ and the parameter $\boldsymbol{\beta}$. Thus, it might be difficult to estimate $\lambda_{0}(t)$ and $\boldsymbol{\beta}$ simultaneously.

The partial likelihood function is
\begin{equation}
    \ell_{p}(\boldsymbol{\beta})=\prod_{j=1}^{m}\frac{\mathrm{e}^{\mathbf{x}_{(j)}^{\prime}\boldsymbol{\beta}}}{\sum_{l\in R\left(t_{(j)}\right)}\mathrm{e}^{\mathbf{x}_{l}^{\prime}\boldsymbol{\beta}}}=\prod_{i=1}^{n}\left[\frac{\mathrm{e}^{\mathbf{x}_{i}^{\prime}\boldsymbol{\beta}}}{\sum_{l\in R\left(t_{i}\right)}\mathrm{e}^{\mathbf{x}_{l}^{\prime}\boldsymbol{\beta}}}\right]^{\delta_{i}}
\end{equation}
where $R(t)$ is the set of individuals who are alive and uncensored at a time just prior to $t_{i}$, which is called the rik set.
