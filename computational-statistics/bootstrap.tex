\chapter{Bootstrap}

Bootstrap is a statistical method for estimating the sampling distribution of an estimator by sampling with replacement from the original sample, most often to derive robust estimates of standard errors and confidence intervals of a population parameter like a mean, median, proportion, odds ratio, correlation coefficient, or regression coefficient.

\section{Bootstrap Principle}

Suppose the i.i.d samples $\left\{\bfx_{1},\bfx_{2},\ldots,\bfx_{n}\right\}$ from an unknown probability distribution $F$ on some probability space $\mcX$. Let $\hat{\theta}_{n}$ be an estimator of $\theta$ based on the sample $\left\{\bfx_{1},\bfx_{2},\ldots,\bfx_{n}\right\}$, that is,
\begin{equation*}
	\hat{\theta}_{n}=s\left(\bfx_{1},\bfx_{2},\ldots,\bfx_{n}\right),
\end{equation*}
where $s(\cdot)$ is some algorithm.

The bootstrap principle is to use the sample $\left\{\bfx_{1},\bfx_{2},\ldots,\bfx_{n}\right\}$ to estimate the sampling distribution of $\hat{\theta}_{n}$.

\paragraph{Nonparametric Bootstrap}

In the $b$-th bootstrap replicate, we sample with replacement from the original sample $\left\{\bfx_{1},\bfx_{2},\ldots,\bfx_{n}\right\}$ to get $\left\{\bfx_{1}^{*b},\bfx_{2}^{*b},\ldots,\bfx_{n}^{*b}\right\}$, and then compute the bootstrap estimate of $\hat{\theta}_{n}$ as
\begin{equation}
	\hat{\theta}_{n}^{*b}=s\left(\bfx_{1}^{*b},\bfx_{2}^{*b},\ldots,\bfx_{n}^{*b}\right).
\end{equation}

\paragraph{Parametric Bootstrap}

\paragraph{Bayesian Bootstrap}

\paragraph{Smooth Bootstrap}

\paragraph{Block Bootstrap}

\section{Standard Error Estimation}

The bootstrap estimate of the standard error of $\hat{\theta}_{n}$ is
\begin{equation}
	\widehat{\operatorname{se}}_{\text{boot}}=\sqrt{\frac{1}{B-1}\sum_{b=1}^{B}\left(\hat{\theta}_{n}^{*b}-\hat{\theta}_{n}^{*}\right)^{2}},
\end{equation}
where $\hat{\theta}_{n}^{*b}$ is the $b$-th bootstrap replicate of $\hat{\theta}_{n}$ and $\hat{\theta}_{n}^{*}$ is the bootstrap estimate of $\hat{\theta}_{n}$, that is,
\begin{equation}
	\hat{\theta}_{n}^{*}=\frac{1}{B}\sum_{b=1}^{B}\hat{\theta}_{n}^{*b}.
\end{equation}

\section{Bias Estimation}

The bootstrap estimate of the bias of $\hat{\theta}_{n}$ is
\begin{equation}
	\widehat{\operatorname{Bias}}_{\text{boot}}=\hat{\theta}_{n}^{*}-\hat{\theta}_{n}.
\end{equation}

\begin{remark}
	The bias-corrected bootstrap estimate of $\hat{\theta}_{n}$ is
	\begin{equation}
		\hat{\theta}_{n}^{*}=\hat{\theta}_{n}-\widehat{\operatorname{Bias}}_{\text{boot}}=2\hat{\theta}_{n}-\hat{\theta}_{n}^{*}.
	\end{equation}
\end{remark}

\section{Confidence Interval Estimation}

\paragraph{Percentile Confidence Interval}

The $1-\alpha$ percentile confidence interval of $\hat{\theta}_{n}$ is
\begin{equation}
	\left[\hat{\theta}_{n}^{*}\left(\alpha/2\right),\hat{\theta}_{n}^{*}\left(1-\alpha/2\right)\right],
\end{equation}
where $\hat{\theta}_{n}^{*}\left(\alpha/2\right)$ is the $\alpha/2$-th percentile of the bootstrap distribution of $\hat{\theta}_{n}$ and $\hat{\theta}_{n}^{*}\left(1-\alpha/2\right)$ is the $\left(1-\alpha/2\right)$-th percentile of the bootstrap distribution of $\hat{\theta}_{n}$.

\paragraph{Bootstrap-t Confidence Interval}

The $1-\alpha$ bootstrap-t confidence interval of $\hat{\theta}_{n}$ is
\begin{equation}
	\left[\hat{\theta}_{n}-t_{n-1}^{*}\left(1-\alpha/2\right)\widehat{\operatorname{se}}_{\text{boot}},\hat{\theta}_{n}-t_{n-1}^{*}\left(\alpha/2\right)\widehat{\operatorname{se}}_{\text{boot}}\right],
\end{equation}
where $t_{n-1}^{*}\left(1-\alpha/2\right)$ is the $\left(1-\alpha/2\right)$-th percentile of the bootstrap distribution of $t_{n-1}$ and $t_{n-1}^{*}\left(\alpha/2\right)$ is the $\alpha/2$-th percentile of the bootstrap distribution of $t_{n-1}$.

\paragraph{Bias-corrected and accelerated (BCa) Confidence Interval}

The $1-\alpha$ bias-corrected and accelerated (BCa) confidence interval of $\hat{\theta}_{n}$ is
\begin{equation}
	\left[\hat{\theta}_{n}^{*}\left(\alpha_{\text{BCa}}/2\right),\hat{\theta}_{n}^{*}\left(1-\alpha_{\text{BCa}}/2\right)\right],
\end{equation}
where $\alpha_{\text{BCa}}$ is the percentile of the bootstrap distribution of $\hat{\theta}_{n}$ that satisfies
\begin{equation}
	\alpha_{\text{BCa}}=\Phi\left(z_{0}+\frac{z_{0}+z_{\alpha}-\widehat{\operatorname{Bias}}_{\text{boot}}}{1-\widehat{\operatorname{a}}}\right),
\end{equation}
where $\Phi(\cdot)$ is the cumulative distribution function of the standard normal distribution, $z_{0}$ is the $1-\alpha/2$ percentile of the standard normal distribution, $z_{\alpha}$ is the $\alpha/2$ percentile of the standard normal distribution, and $\widehat{\operatorname{a}}$ is the percentile of the bootstrap distribution of $\hat{\theta}_{n}$ that satisfies
\begin{equation}
	\widehat{\operatorname{a}}=\frac{\sum_{b=1}^{B}\left(\hat{\theta}_{n}^{*b}-\hat{\theta}_{n}^{*}\right)^{3}}{6\left[\sum_{b=1}^{B}\left(\hat{\theta}_{n}^{*b}-\hat{\theta}_{n}^{*}\right)^{2}\right]^{3/2}}.
\end{equation}

\section{Hypothesis Testing}

\paragraph{One-sample Hypothesis Testing}

\paragraph{Two-sample Hypothesis Testing}

\section{Jackknife}

Let $\bfx_{-i}$ be the sample with $x_{i}$ removed, $\bfx_{-i}=\left(x_{1},x_{2},\ldots,x_{i-1},x_{i+1},\ldots,x_{n}\right)^{\top}$, and denote the corresponding value of the statistic of interest as
\begin{equation}
	\hat{\theta}_{-i}=s\left(\bfx_{-i}\right)
\end{equation}

\paragraph{Bias of $\hat{\theta}$}

For almost all reasonable and practical estimates, we have
\begin{equation*}
	\operatorname{Bias}(\hat{\theta}_{n})=\bbE(\hat{\theta}_{n})-\theta\rightarrow 0,\quad n\rightarrow\infty
\end{equation*}
Then, it is reasonable to assume a power series of the type
\begin{equation*}
	\bbE(\hat{\theta}_{n})=\theta+\frac{a_{1}}{n}+\frac{a_{2}}{n^{2}}+\frac{a_{3}}{n^{3}}+\ldots
\end{equation*}
with some coefficients $\left\{a_{k}\right\}$. And we have
\begin{equation*}
	\bbE(\hat{\theta}_{-i})=\bbE(\hat{\theta}_{n-1})=\theta+\frac{a_{1}}{n-1}+\frac{a_{2}}{\left(n-1\right)^{2}}+\frac{a_{3}}{\left(n-1\right)^{3}}+\ldots
\end{equation*}

For the sake of a smaller variance, we average all such estimates and let
\begin{equation*}
	\hat{\theta}_{(\cdot)}=\frac{1}{n}\sum_{i=1}^{n}\hat{\theta}_{-i}
\end{equation*}
thus,
\begin{equation*}
	\bbE(\hat{\theta}_{(\cdot)})=\theta+\frac{a_{1}}{n-1}+\frac{a_{2}}{\left(n-1\right)^{2}}+\frac{a_{3}}{\left(n-1\right)^{3}}+\ldots
\end{equation*}

Thus, we have
\begin{equation*}
	(n-1)\bbE\left[\hat{\theta}_{(\cdot)}-\hat{\theta}_{n}\right]=\frac{a_{1}}{n}+\frac{a_{2}}{n^{2}}+\frac{a_{3}}{n^{3}}+\ldots=\operatorname{Bias}(\hat{\theta})
\end{equation*}
Hence, we can get the jackknife estimate bias for $\hat{\theta}$ be
\begin{equation}
	\widehat{\operatorname{Bias}}_{\text{jack}}=(n-1)\left(\hat{\theta}_{(\cdot)}-\hat{\theta}_{n}\right)
\end{equation}

% Variance of the bias of $\hat{\theta}$

\begin{remark}
	It is easy to combine the averaged Jackknife estimator $\hat{\theta}_{-i}$ with the original $\hat{\theta}$, to kill the main term in the bias of $\hat{\theta}$, thus,
	\begin{equation*}
		\begin{aligned}
			\bbE\left[n\hat{\theta}_{n}-(n-1)\hat{\theta}_{(\cdot)}\right] & =\left[n\theta-(n-1)\theta\right]+\left[a_{1}-a_{1}\right]+\left[\frac{a_{2}}{n}-\frac{a_{2}}{n-1}\right]+\ldots \\
			                                                               & =\theta+\frac{a_{2}}{n(n-1)}+\ldots=\theta+\frac{a_{2}}{n^{2}}+O\left(n^{-3}\right)
		\end{aligned}
	\end{equation*}
	This removes the bias in the special case that the bias is $O\left(n^{-1}\right)$ and removes it to $O\left(n^{-2}\right)$ in other cases.
\end{remark}

\paragraph{Variance of $\hat{\theta}$}

The jackknife estimate of variance for $\hat{\theta}$ is
\begin{equation}
	\widehat{\Var}_{\text{jack}}=\frac{n-1}{n}\sum_{i=1}^{n}\left(\hat{\theta}_{-i}-\hat{\theta}_{(\cdot)}\right)^{2},\quad\text{where }\hat{\theta}_{(\cdot)}=\frac{1}{n}\sum_{i=1}^{n}\hat{\theta}_{-i}
\end{equation}

The jackknife method of estimation can fail if the statistic $\hat{\theta}_{\text{jack}}$ is not smooth. Smoothness
implies that relatively small changes to data values will cause only a small change in the
statistic.

\begin{example}[Sample Mean]

\end{example}

\begin{example}[Sample Correlation Coefficient]

\end{example}
