\chapter{Lebesgue Integration}

\section{Properties of the Integral}

\begin{theorem}[Jensen's Inequality]
	Let $(\Omega,A,\mu)$ be a probability space. If $f$ is a real-valued function that is $\mu$-integrable, and if $\varphi$ is a convex function on the real line, then:
	\begin{equation}
		\varphi\left(\int_{\Omega}f\dif\mu\right)\leq\int_{\Omega}\varphi(f)\dif\mu.
	\end{equation}
\end{theorem}

\begin{proof}
	Let $x_{0}=\int_{\Omega}f\dif\mu$. Since the existence of subderivatives for convex functions, $\exists a,b\in R$, such that,
	\begin{equation*}
		\forall x\in R,\varphi(x)\geq ax+b\text{ and }ax_0+b=\varphi(x_0).
	\end{equation*}
	Then, we got
	\begin{equation*}
		\int_{\Omega}\varphi(f)\dif\mu\geq\int_{\Omega}af+b\dif\mu=a\int_{\Omega}f\dif\mu+b=ax_0+b=\varphi\left(\int_{\Omega}f\dif\mu\right).
	\end{equation*}
\end{proof}

\begin{theorem}[H\"older's Inequality] \label{thm:holder-inequality}
	Let $(\Omega,\mcF,\mu)$ be a measure space and let $p,q\in[1,\infty]$ with $1/p+1/q=1$. Then, for all measurable functions $f$ and $g$ on $\Omega$,
	\begin{equation}
		\int_{\Omega}|f\cdot g|\dif\mu\leq\|f\|_{p}\|g\|_{q}.
	\end{equation}
	% If, in addition, $p,q\in(1,\infty)$ and $f\in L^{p}(\mu)$ and $g \in L^{q}(\mu)$, then Hölder's inequality becomes an equality iff $|f|^{p}$ and $|g|^{q}$ are linearly dependent in $L^{1}(\mu)$, meaning that there exist real numbers $\alpha, \beta \geq 0,$ not both of them zero, such that $\alpha|f|^{p}=\beta|g|^{q} \mu$ -almost everywhere.
\end{theorem}

\begin{proof}

\end{proof}

\begin{theorem}[Minkowski's Inequality] \label{thm:minkowski-inequality}
	Let $(\Omega,\mcF,\mu)$ be a measure space and let $p\in[1,\infty]$. Then, for all measurable functions $f$ and $g$ on $\Omega$,
	\begin{equation}
		\|f+g\|_{p} \leq\|f\|_{p}+\|g\|_{p}.
	\end{equation}
\end{theorem}

\begin{proof}
	Since $\varphi(x)=x^p$ is a convex function for $p\in[1,\infty)$. By it's definition,
	\begin{equation*}
		|f+g|^{p}=\left|2\cdot\frac{f}{2}+2\cdot\frac{g}{2}\right|^{p}\leq \frac{1}{2}|2f|^p+\frac{1}{2}|2g|^p=2^{p-1}\left(|f|^{p}+|g|^{p}\right).
	\end{equation*}
	Therefore,
	\begin{equation*}
		|f+g|^{p}<2^{p-1}\left(|f|^{p}+|g|^{p}\right)<\infty.
	\end{equation*}
	By H\"older's Inequality (\ref{thm:holder-inequality}),
	\begin{equation*}
		\begin{aligned}
			\|f+g\|_{p}^{p} & =\int|f+g|^{p}\dif\mu                                                                                                                                                                             \\
			                & =\int|f+g| \cdot|f+g|^{p-1}\dif \mu                                                                                                                                                               \\
			                & \leq \int(|f|+|g|)|f+g|^{p-1}\dif\mu                                                                                                                                                              \\
			                & =\int|f||f+g|^{p-1}\dif\mu+\int|g||f+g|^{p-1}\dif\mu                                                                                                                                              \\
			                & \leq\left(\left(\int|f|^{p} \dif \mu\right)^{\frac{1}{p}}+\left(\int|g|^{p} \dif \mu\right)^{\frac{1}{p}}\right)\left(\int|f+g|^{(p-1)\left(\frac{p}{p-1}\right)} \dif \mu\right)^{1-\frac{1}{p}} \\
			                & =\left(\|f\|_{p}+\|g\|_{p}\right) \frac{\|f+g\|_{p}^{p}}{\|f+g\|_{p}}
		\end{aligned}
	\end{equation*}
	which means, as $p\in[1,\infty)$,
	\begin{equation*}
		\|f+g\|_{p} \leq\|f\|_{p}+\|g\|_{p}.
	\end{equation*}
	When $p=\infty$,
	\begin{equation*}
		a
	\end{equation*}
\end{proof}

\begin{theorem}[Bounded Convergence Theorem]

\end{theorem}

\begin{theorem}[Fatou's Lemma]

\end{theorem}

\begin{theorem}[Monotone Convergence Theorem]

\end{theorem}

\section{Product Measures}

\begin{theorem}[Fubini's Theorem]

\end{theorem}
