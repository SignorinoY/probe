\chapter{Measure Theory}

\section{Semi-algebras, Algebras and Sigma-algebras}

\begin{definition}[Semi-algebra]
    A nonempty class of $\mathcal{S}$ of subsets of $\Omega$ is an \textbf{semi-algebra} on $\Omega$ that satisfy
    \begin{enumerate}
        \item if $A,B\in\mathcal{S}$, then $A\cap B\in\mathcal{S}$.
        \item if $A\in\mathcal{S}$, then $A^C$ is a finite disjoint union of sets in $\mathcal{S}$, i.e., $$A^C=\sum_{i=1}^{n}A_i, \text{where} A_i\in\mathcal{S}, A_i\cap A_j=\emptyset ,i\neq j.$$
    \end{enumerate}
\end{definition}

\begin{definition}[Algebra]
    A nonempty class $\mathcal{A}$ of subsets of $\Omega$ is an \textbf{algebra} on $\Omega$ that satisfy
    \begin{enumerate}
        \item if $A\in\mathcal{A}$, then $A^C\in\mathcal{A}$.
        \item if $A_1, A_2\in\mathcal{A}$, then $A_1\cup A_2\in\mathcal{A}$.
    \end{enumerate}
\end{definition}

% \begin{theorem}
%     If $\mathcal{A}$ is an algebra (or a $\sigma$-algebra), then $\emptyset\in\mathcal{A}$ and $\Omega\in\mathcal{A}$. However, the same may not hold for semi-algebras.
% \end{theorem}

% \begin{theorem}
%     $\mathcal{A}$ is an algebra $\Longleftrightarrow$ $\Omega\in\mathcal{A}$; if $A,B\in\mathcal{A}$, then $A-B\in\mathcal{A}$. 
% \end{theorem}

\begin{definition}[$\sigma$-algebra]
    A nonempty class $\mathcal{F}$ of subsets of $\Omega$ is a \textbf{$\sigma$-algebra} on $\Omega$ that satisfy
    \begin{enumerate}
        \item if $A\in\mathcal{F}$, then $A^C\in\mathcal{F}$.
        \item if $A_i\in\mathcal{F}$ is a countable sequence of sets, then $\cup_iA_i\in\mathcal{F}$.
    \end{enumerate}
\end{definition}

% \begin{proposition}{The relationship between semi-algebras, algebras and $\sigma$-algebras]
%     \begin{enumerate}
%         \item A semi-algebra may not be an algebra.
%         \item An algebra may not be  a $\sigma$-algebra.
%     \end{enumerate}
% \end{proposition}

% \begin{proof}

% \end{proof}

\begin{example}[ Special $\sigma$-algebra]
    \begin{enumerate}
        \item \textbf{Trival $\sigma$-algebra} $:=\{\emptyset,\Omega\}$. This is smallest $\sigma$-algebra.
        \item \textbf{Power Set} $:=$ all subsets of $\sigma$, denoted by $\mathcal{P}(\Omega)$. This is the largest $\sigma$-algebra.
        \item \textbf{The smallest $\sigma$-algebra containing $A\in\Omega$} $:=\{\emptyset,A,A^C,\Omega\}$.
    \end{enumerate}
\end{example}

It is easy to define (Lesbegue) measure on the semi-algebra  $\mathcal{S}$, and then easily to extend it to the algebra $\overline{\mathcal{S}}$, finally, we can extend it further t o some $\sigma$-algebra (mostly consider the smallest one containing $\mathcal{S}$).

\begin{lemma}
    If $\mathcal{S}$ is a semi-algebra, then $$\overline{\mathcal{S}}=\{\text{finite disjoint unions of sets in }\mathcal{S}\}$$ is an algebra, denoted by $\mathcal{A}(\mathcal{S})$, called \textbf{the algebra generated by $\mathcal{S}$}.
\end{lemma}

\begin{proof}
    Let $A,B\in\overline{\mathcal{S}}$, then $A=\sum_{i=1}^{n}A_i, B=\sum_{j=1}^{m}B_j$ with $A_i,B_i\in\mathcal{S}$.\par
    \textbf{Intersection}: For $A_i\cap B_j\in\mathcal{S}$ by the definition of semi-algebra $\mathcal{S}$, thus $$A\cap B=\sum_{i=1}^{n}\sum_{j=1}^{m}A_i\cap B_j\in\overline{\mathcal{S}}.$$ So $\overline{\mathcal{S}}$ is closed under (finite) intersection.\par
    \textbf{Complement}: For DeMorgan's Law, $A_i^C\in\mathcal{S}$ by the definition of semi-algebra $\mathcal{S}$ and $\overline{\mathcal{S}}$ closed under (finite) intersection that we just shown, thus $$A^C=(\sum_{i=1}^{n}A_i)^C=\cap_{i=1}^{n}A_i^C\in\overline{\mathcal{S}}.$$ So $\overline{\mathcal{S}}$ is closed under complement.\par
    \textbf{Union}: For DeMorgan's Law and $\overline{\mathcal{S}}$ closed under (finite) intersection and complement that we just shown, thus $$A\cup B=(A^C\cap B^C)^C\in\overline{\mathcal{S}}.$$ So $\overline{\mathcal{S}}$ is closed under (finite) union.\par
    Hence, $\overline{\mathcal{S}}$ is an algebra.
\end{proof}

\begin{theorem}
    For any class $\mathcal{A}$, there exists a unique minimal $\sigma$-algebra containing $\mathcal{A}$, denoted by $\sigma(\mathcal{A})$, called \textbf{the $\sigma$-algebra generated by $\mathcal{A}$}. In other words,
    \begin{enumerate}
        \item $\mathcal{A}\subset\sigma(\mathcal{A})$.
        \item For any $\sigma$-algebra $\mathcal{B}$ with $\mathcal{A}\subset\mathcal{B}$, $\sigma(\mathcal{A})\subset\mathcal{B}$.
    \end{enumerate}
    and $\sigma(\mathcal{A})$ is unique.
\end{theorem}

\begin{proof}
    \textbf{Existence}:\par
    \textbf{Uniqueness}:\par
\end{proof}

\begin{example}[ Borel $\sigma$-algebras generated from semi-algebras]
    \begin{enumerate}
        \item
    \end{enumerate}
    % The $\sigma$-algebra generated by the collection of all open intervals on the real line $\mathcal{R}=(-\infty,\infty)$ is called the \textbf{Borel $\sigma$-algebra}, denoted by $\mathcal{B}$. The elements of $\mathcal{B}$ are called \textbf{Borel sets}.
\end{example}

\section{Measure}

\begin{definition}[Measure]
    \textbf{Measure} is a nonnegative countably additive set function, that is, a function $\mu:\mathcal{A}\rightarrow\mathbf{R}$ with
    \begin{enumerate}
        \item $\mu(A)\geq\mu(\emptyset)=0$ for all $A\in\mathcal{A}$.
        \item if $A_i\in\mathcal{A}$ is a countable sequence of disjoint sets, then $$\mu(\cup_iA_i)=\sum_i\mu(A_i).$$
    \end{enumerate}
\end{definition}

% \section{Measure Space}

% \begin{definition}[Probability Space]
%     \textbf{Probability space} is a triple $(\Omega,\mathcal{F}, P)$, which consist of three elements:
%     \begin{enumerate}
%         \item A sample space $\Omega$, which is the arbitray non-empty set of all possible outcomes.
%         \item An event space $\mathcal{F}$, which is a set of events, an event being a set of outcomes in the sample space. We assume that $\mathcal{F}$ is a $\sigma$-field (or $\sigma$-algebra),
%         \item A probability measure $P:\mathcal{F}\rightarrow[0,1]$, which assigns each event in the event space a probability.
%     \end{enumerate}
% \end{definition}

% \begin{note}
%     Without $P$, $(\Omega, \mathcal{F})$ is called a \textbf{measureable space}, i.e., it is a space on which we can put a measure.
% \end{note}

% \begin{note}
%     If $\mu(\Omega)=1$, we call $\mu$ a \textbf{probability measure}, which are usually denoted by $P$.
% \end{note}

\begin{definition}[Measure Space]
    If $\mu$ is a measure on a $\sigma$-algebra $\mathcal{A}$ of subsets of $\Omega$, the triplet $(\Omega,\mathcal{A},\mu)$ is a \textbf{measure space}.
\end{definition}

\begin{note}
    A measure space $(\Omega,\mathcal{A},\mu)$ is a \textbf{probability space}, if $P(\Omega)=1$.
\end{note}

\begin{property}
    Let $\mu$ be a measure on a $\sigma$-algebra $\mathcal{A}$
    \begin{enumerate}
        \item \textbf{monotonieity} if $A\subset B$, then $\mu(A)\leq\mu(B)$.
        \item \textbf{subadditivity} if $A\subset\cup_{m=1}^{\infty} A_m$, then $\mu(A)\leq\sum_{m=1}^{\infty}  u(A_m)$.
        \item \textbf{continuity from below} if $A_i\uparrow A$ (i.e. $A_1\subset A_2\subset \ldots$ and $\cup_iA_i=A$), then $\mu(A_i)\uparrow \mu(A)$.
        \item \textbf{continuity from above} if $A_i\downarrow A$ (i.e. $A_1\supset A_2\supset \ldots$ and $\cap_iA_i=A$), then $\mu(A_i)\downarrow \mu(A)$.
    \end{enumerate}
\end{property}

\begin{proof}

\end{proof}
