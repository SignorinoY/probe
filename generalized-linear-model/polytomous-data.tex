\chapter{Polytomous Data}

\begin{definition}[Polytomous Data]
    A response is polytomous, if the response of an individual or item in a study is \textbf{restricted to one of a fixed set of possible values}.
\end{definition}

\begin{remark}
    There are two types of scales, pure scales and compound scales \footnote{A bivariate responses with one response ordinal and the other continuous is an example of compound scales.}. For pure scales, there are several types:
    \begin{enumerate}
        \item \textbf{Nominal Scale}: a scale used for labeling variables into distinct classifications and does not involve a quantitative value or order.
        \item \textbf{Ordinal Scale}: a variable measurement scale used to simply depict the order of variables and not the difference between each of the variables.
        \item \textbf{Interval Scale}: a numerical scale where the order of the variables is known as well as the difference between these variables.
    \end{enumerate}
\end{remark}

\section{Model Assumption}

Let the category probabilities given $\mathbf{x}_{i}$ be
\begin{equation}
    \pi_{j}\left(\mathbf{x}_{i}\right)=P\left(Y=y_{j}\mid\mathbf{X}=\mathbf{x}_{i}\right)
\end{equation}
and the cumulative probabilities given $\mathbf{x}_{i}$ be
\begin{equation}
    r_{j}\left(\mathbf{x}_{i}\right)=P\left(Y\leq\sum_{r\leq j}y_{r}\mid\mathbf{X}=\mathbf{x}_{i}\right)
\end{equation}
where $i=1,2,\ldots,n,\quad j=1,2,\ldots,k$.

\subsection{Response Distribution}

The multinomial distribution is in many ways the most natural distribution to consider in the context of a polytomous response variable.

\subsection{Link Function}

\subsubsection*{Nominal Scale}

\begin{equation}
    \pi_{j}\left(\mathbf{x}_{i}\right)=\frac{\exp \left[\eta_{j}\left(\mathbf{x}_{i}\right)\right]}{\sum_{j=1}^{k} \exp \left[\eta_{j}\left(\mathbf{x}_{i}\right)\right]}
\end{equation}
where $\eta_{j}\left(\mathbf{x}_{i}\right)=\eta_{j}\left(\mathbf{x}_{0}\right)+\left(\mathbf{x}_{i}-\mathbf{x}_{0}\right)^{\prime}\boldsymbol{\beta}_{j}+\alpha_{i}$.

\subsubsection*{Ordinal Scale}

\begin{enumerate}
    \item Logistic Scale:
          \begin{equation}
              \log\left[\frac{r_{j}\left(\mathbf{x}_{i}\right)}{1-r_{j}\left(\mathbf{x}_{i}\right)}\right]=\theta_{j}-\mathbf{x}_{i}^{\prime}\boldsymbol{\beta}
          \end{equation}
    \item Complementary Log-Log Scale:
          \begin{equation}
              \log\left\{-\log\left[1-r_{j}\left(\mathbf{x}_{i}\right)\right]\right\}=\theta_{j}-\mathbf{x}_{i}^{\prime}\boldsymbol{\beta}
          \end{equation}
\end{enumerate}

\subsubsection*{Interval Scale}

Suppose the $j$-th category exits a cardinal number or score, $s_j$, where the difference between scores is a measure of distance between or separation of categories.

\begin{enumerate}
    \item \begin{equation}
              \log\left[\frac{r_{j}\left(\mathbf{x}_{i}\right)}{1-r_{j}\left(\mathbf{x}_{i}\right)}\right]=\varsigma_{0}+\varsigma_{1}\left(\frac{s_{j}+s_{j+1}}{2}\right)-\mathbf{x}_{i}^{\prime}\boldsymbol{\beta}-\mathbf{x}\xi\left(c_{j}-\bar{c}\right)
          \end{equation}
          where $c_{j}=\frac{s_{j}+s_{j+1}}{2}$ or $c_{j}=\operatorname{logit}\left(\frac{s_{j}+s_{j+1}}{2}\right)$.
    \item \begin{equation}
              \pi_{j}\left(\mathbf{x}_{i}\right)=\frac{\exp \left[\eta_{j}\left(\mathbf{x}_{i}\right)\right]}{\sum_{j=1}^{k} \exp \left[\eta_{j}\left(\mathbf{x}_{i}\right)\right]}
          \end{equation}
          where $\eta_{j}\left(\mathbf{x}_{i}\right)=\eta_{j}+\left(\mathbf{x}_{i}\boldsymbol{\beta}\right)s_{j}+\alpha_{i}$.
    \item \begin{equation}
        \sum_{j=1}^{k}\pi_{j}\left(\mathbf{x}_{i}\right)s_{j}=\mathbf{x}_{i}\boldsymbol{\beta}
    \end{equation}
\end{enumerate}