\chapter{Unconstrained Minimization}

\section{Definition of Unconstrained Minimization}

\begin{definition}[Unconstrained Minimization Problem]
	The unconstrained minimization problem is defined to be
	\begin{equation}
		\min_{\bfx}f(\bfx)
		\label{eq:unconstrained-minimization-problem}
	\end{equation}
	where $f:\bbR^n\rightarrow\bbR$ is convex and twice continuously differentiable.
\end{definition}

Assume the problem is solvable, i.e., there exists an optimal point $\bfx^{*}$, such that,
\begin{equation*}
	f(\bfx^{*})=\inf_{\bfx}f(\bfx)
\end{equation*}
and denote it by $p^{*}$. Since $f$ is differentiable and convex, the point $\bfx^{*}$ is optimal. if and only if
\begin{equation}
	\nabla f(\bfx^{*})=0
	\label{eq:optimal-condition-of-unconstrained-minimization-problem}
\end{equation}

Solving (\ref{eq:unconstrained-minimization-problem}) is equal to finding the solution of (\ref{eq:optimal-condition-of-unconstrained-minimization-problem}), thus (\ref{eq:unconstrained-minimization-problem}) can be solved by analytic solution of (\ref{eq:optimal-condition-of-unconstrained-minimization-problem}) in a few cases, but usually can be solved by an iterative algorithm, i.e.,
\begin{equation*}
	\exists\bfx^{(0)},\bfx^{(1)},\ldots\in\operatorname{dom}f,\quad\text{ s.t. }f(\bfx^{(k)})\rightarrow p^{*},\quad\text{ as }\quad k\rightarrow\infty
\end{equation*}
This algorithm is terminated when $f\left(x^{(k)}\right)-p^{\star}\leq\epsilon$, where $\epsilon>0$ is some specified tolerance.

\begin{remark}
	The initial point $\bfx^{(0)}$ must lie in $\operatorname{dom}f$, and the sublevel set
	\begin{equation*}
		S=\left\{\bfx\in\operatorname{dom}f\mid f(\bfx)\leq f(\bfx^{(0)})\right\}
	\end{equation*}
	must be closed. Any closed function (Definition \ref{def:closed-function})
\end{remark}

\begin{example}[Quadratic Minimization]
	The general convex quadratic minimization problem has the form
	\begin{equation}
		\min_{\bfx}\,\frac{1}{2}\bfx^{\top}\bfP\bfx+\bfq^{\prime}\bfx+r \label{eq:quadratic-minimization}
	\end{equation}
	where $\bfP\in\mathbb{S}_{+}^{n},\bfq\in\bbR^{n}$, and $r\in\bbR$. The optimality condition is
	\begin{equation}
		\bfP\bfx^{*}+\bfq=\bfzero
		\label{eq:quadratic-minimization-optimality-condition}
	\end{equation}
	which is a set of linear equations.
	\begin{enumerate}
		\item If $\bfP\succ 0$, exists a unique solution $\bfx^{*}=-\bfP^{-1}\bfq$.
		\item If $\bfP$ is not positive definite, any solution of (\ref{eq:quadratic-minimization-optimality-condition}) is optimal for (\ref{eq:quadratic-minimization}).
		\item If (\ref{eq:quadratic-minimization-optimality-condition}) does not have a solution, then (\ref{eq:quadratic-minimization}) is unbounded.
	\end{enumerate}
\end{example}

\begin{proof}
	\hfill
	\begin{enumerate}
		\item
		      Obviously.

		\item
		      Since $\bfP\nsucceq 0$, i.e.,
		      \begin{equation*}
			      \exists\bfv,\quad\text{ s.t. }\bfv^{\prime}\bfP\bfv<0
		      \end{equation*}
		      Let $\bfx=t\bfv$, we have
		      \begin{equation*}
			      f\left(\bfx\right)=t^{2}\left(\bfv^{\prime}\bfP\bfv/2\right)+t\left(\bfq^{\prime}\bfv\right)+r
		      \end{equation*}
		      which converges to $-\infty$ as $t\rightarrow\infty$.

		\item
		      Since (\ref{eq:quadratic-minimization-optimality-condition}) does not have a solution, i.e.,
		      \begin{equation*}
			      \bfq\notin\mathcal{R}(\bfP)
		      \end{equation*}
		      Let
		      \begin{equation*}
			      \bfq=\tilde{\bfq}+\bfv
		      \end{equation*}
		      where $\tilde{\bfq}$ is the Euclidean projection of $\bfq$ onto $\mathcal{R}(\bfP)$, and $\bfv=\bfq-\tilde{\bfq}$. And $\bfv$ is nonzero and orthogonal to $\mathcal{R}(\bfP)$, i.e., $\bfv^{\prime}\bfP\bfv=0$. If we take $\bfx=t\bfv$, we have
		      \begin{equation*}
			      f(\bfx)=t\bfq^{\prime}\bfv+r=t(\tilde{\bfq}+\bfv)^{\prime}\bfv+r=t(\bfv^{\prime}\bfv)+r
		      \end{equation*}
		      which is unbounded below.
	\end{enumerate}
\end{proof}

\begin{remark}
	The least-squares problem is a special case of quadratic minimization, that,
	\begin{equation*}
		\min_{\bfx}\,\|\bfA\bfx-\bfb\|_{2}^{2}=\bfx^{\top}\left(\bfA^{\prime}\bfA\right)\bfx-2\left(\bfA^{\prime}\bfb\right)^{\prime}\bfx+\bfb^{\prime}\bfb
	\end{equation*}
	The optimality condition is
	\begin{equation*}
		\bfA^{\prime}\bfA\bfx^{*}=\bfA^{\prime}\bfb
	\end{equation*}
	are called the normal equations of the least-squares problem.
\end{remark}

\begin{example}[Unconstrained Geometric Programming]
	The unconstrained geometric program in convex form
	\begin{equation*}
		\min_{\bfx}\,f(\bfx)=\log \left(\sum_{i=1}^{m}\exp\left(\bfa_{i}^{\prime}\bfx+b_{i}\right)\right)
	\end{equation*}
	The optimality condition is
	\begin{equation*}
		\nabla f\left(x^{*}\right)=\frac{\sum_{i=1}^{m}\exp\left(\bfa_{i}^{\prime}\bfx^{*}+b_{i}\right)\bfa_{i}}{\sum_{j=1}^{m}\exp\left(\bfa_{j}^{\prime}\bfx^{*}+b_{j}\right)}=\bfzero
	\end{equation*}
	which has no analytical solution, so we must resort to an iterative algorithm. For this problem, $\operatorname{dom} f=\bbR^{n}$, so any point can be chosen as the initial point $\bfx^{(0)}$.
\end{example}

\begin{example}[Analytic Center of Linear Inequalities]
	Consider the optimization problem
	\begin{equation*}
		\min_{\bfx}\,f(x)=-\sum_{i=1}^{m}\log\left(\bfb_{i}-\bfa_{i}^{T}\bfx\right)
	\end{equation*}
	where the domain of $f$ is the open set
	\begin{equation*}
		\operatorname{dom}f=\left\{\bfx\mid\bfa_{i}^{\prime}\bfx<\bfb_{i},i=1,\ldots,m\right\}
	\end{equation*}
\end{example}

\begin{definition}[Strong Convexity]

\end{definition}

\section{General Descent Method}

\section{Gradient Descent Method}

\section{Steepest Descent Method}

\section{Newton's Method}

 (Smoothness Hessian) Suppose the Hessians of $f$ are Lipschitz continuous, i.e.,
\begin{equation}
	\|\nabla^{2}f(\bfx)-\nabla^{2}f(\bfy)\|_{2}\leq L\|\bfx-\bfy\|_{2}
	\label{eq:smoothness-hessian}
\end{equation}

\begin{algorithm}[htbp]
	\caption{Damped Newton Method}
	\KwIn{Initial point $\bfx_0\in\dom f$, tolerance $\epsilon>0$}
	\KwOut{}
	\Repeat{}{
		Compute the Newton step and decrement
		\begin{equation*}
			\Delta\bfx_{\text{nt}}:=-\nabla^{2}
		\end{equation*}
	}
\end{algorithm}

\begin{theorem}
	Under the condition , there exist $0<\eta<m^{2}/L$ and $\gamma>0$, for the damped Newton method, we have
	\begin{itemize}
		\item If $\|\nabla^{2}f(\bfx^{(k)})\|\geq\eta$, then
		      \begin{equation*}
			      f(\bfx^{(k+1)})-f(\bfx^{(k)})\leq-\gamma.
		      \end{equation*}
		\item If $\|\nabla^{2}f(\bfx^{(k)})\|<\eta$, then the backtracking line search select $t^{(k)}=1$, and
		      \begin{equation*}
			      \frac{L}{2m^{2}}\|\nabla f(\bfx^{(k+1)})\|_{2}\leq \left(\frac{L}{2m^{2}}\|\nabla f(\bfx^{(k)})\|_{2}\right)^{2}.
		      \end{equation*}
	\end{itemize}
\end{theorem}

\begin{table}[htbp]
	\centering
	\begin{tabular}{cccc}
		\toprule
		Method & Descent Direction & Step Length & Features \\
		\midrule
		Steepest                                            \\
		Steepest (MG)                                       \\
		Steepest (CD)                                       \\
		Steepest (BB)                                       \\
		\midrule
		Newton                                              \\
		Newton (LM)                                         \\
		Newton (Mixed)                                      \\
		\midrule
		Quasi-Newton (SR1)                                  \\
		Quasi-Newton (DFP)                                  \\
		Quasi-Newton (BFGS)                                 \\
		Quasi-Newton (LBFGS)                                \\
		\bottomrule
	\end{tabular}
\end{table}

\begin{example}[Extended Rosenbrock Function]
	\begin{equation}
		\min_{\bfx}\,f(\bfx)=\sum_{i=1}^{n}r_{i}^{2}(\bfx)
	\end{equation}
	where $n$ is even, and
	\begin{equation}
		r_{i}(\bfx)=\begin{cases}
			10(x_{2k}-x_{2k-1}^{2}), & i=2k-1 \\
			1-x_{2k-1},              & i=2k   \\
		\end{cases}
	\end{equation}
	The minimum point is $\bfx^{*}=(1,1,\ldots,1)^{\prime}$, the initial point is $\bfx_{0}=(-1.2,1,\ldots,-1.2,1)^{\prime}$.
\end{example}
